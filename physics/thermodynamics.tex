\documentclass[a4paper]{jsreport}
\usepackage[utf8]{inputenc}
\usepackage{amsmath}

\title{熱力学 - Thermodynamics}

\begin{document}

    \maketitle

    \tableofcontents

    \chapter{熱力学とは}
        \section{熱力学という学問}
        このPDFがあらかた完成してから書くと思う
            
        \section{熱力学的現象の見方}

            \subsection*{仕事}
                熱力学では、力学的な操作と紐付けた理論が構築されている。
                例えば、系に対して仕事をすれば内部エネルギーが上昇する現象などのようなものがそれに該当する。\par
                唐突ではあるが、天下り的に状態方程式について考える。力学的な仕事に直接的な関係がある圧力$P$について解くと、
                \begin{equation}
                    p = \frac{NRT}{V}
                \end{equation}
                となる。これは理想機体に適用される式なので、熱力学的現象全般に対して適用できるものではないが、力学的な量(今回は圧力$P$)は体積$V$、物質量$N$、絶対温度$T$が密接に関わっていることを直感的に理解してもらえると思う。
                ここから先、系の状態を記述する際にキーとなるパラメータは、ここで挙げた3つになる。

            \subsection*{示量変数と示強変数}
                体積$V$や物質量$N$のような物理量は、その熱力学的な系の\textbf{量}を示すパラメータであるため\textbf{示量変数}と呼ばれる。
                それに対して、絶対温度$T$のような物理量は、系の\textbf{強度}を示すパラメータであるため\textbf{示強変数}と呼ばれる。\par
                熱力学的な系の状態を表現する記法として、以後$(T; V, N)$や$(T; X),X=(V, N)$のようにセミコロンの左側に示強変数、右側に示量変数を並べるような記法で表現することにする。\par
                先ほど挙げた2つの変数の違いを直感的に理解するために具体例をひとつ提示する。
                お湯と冷えた水を用意し、お湯は$(T_1; V_1, N_1)$、水は$(T_2; V_2, N_2)$とする。
                ここで、2つの液体を混ぜ合わせるとそれぞれの物理量はどうなるだろうか。
                \begin{equation}
                    V = V_1 + V_2
                \end{equation}   
                \begin{equation}
                    N = N_1 + N_2
                \end{equation}   
                \begin{equation}
                    T \ne T_1 + T_2
                \end{equation}   
                体積$V$や物質量$N$のような示量変数のみ相加性を持つことがわかる。
                この違いは現象を解析していく上で、大変重要な違いである。
            
            \subsection*{平衡状態}
                平衡状態を簡潔に表現すると、自発的な状態の変化がなくなった状態のことである。
                熱力学では、平衡状態間での変化を取り扱うため、考えている系が平衡かどうかをチェックすることは非常に重要である。 

            \subsection*{最大仕事}
                次のチャプターでも言及するが、仕事の値は始状態と終状態が同じだとしても一意に定まるわけではない。
                外界から仕事をすることで、$(T; X)\xrightarrow{}(T'; X')$に状態を変化させるとしても、そのさせ方によって終状態は変わってくる。
                イメージとしては、腕立てを一回するという動作について考えてみたいのだが、「腕を伸ばした状態から腕を曲げることで体を落としていき、その後元の状態に戻す」表現しておく。
                始状態と終状態は同じ腕を伸ばした状態であるが、腕立て一回の速さによって腕への負担は変わってくる。
                つまり、変化のさせ方によって仕事の値は変わってきてしまうということである。

            \subsection*{準静的過程}   
                系の状態を変化させる際に、その変化速度が極限的に$0$の場合、その操作のことを\textbf{準静的操作}という。
                噛み砕いて言うと、極限までゆっくり変化させていくということである。
                これはすなわち、平衡状態を保ったまま変化させることに対応し、理論を組み立てていく上で重要な仮定になる。

    \chapter{等温操作}

        \section{等温操作とは}
            \subsection*{等温操作}
                等温操作とは、系が温度一定の環境下で変化させることである。
                つまり、以下のような温度が一定の操作のことである。
                矢印上部の$i$は等温操作であることを表している。ちなみに、$iq$で等温準静操作を表す。
                \begin{equation}
                    (T; X)\xrightarrow{i}(T; X')
                \end{equation}
                ただひとつ注意として、環境が等温なだけで、考えている系が常に等温なわけではない。
                始状態と終状態が環境の温度と等しければ、それは等温操作となる。
                
        \section{ケルヴィンの定理}
                               
            \subsection*{ケルヴィンの定理(熱力学第二法則)}
                ある状態$(T; X)$から状態を変化させていき、最終的に$(T; X)$に帰ってくるような\textbf{等温サイクル}を考える。
                このサイクルで系が外部にする仕事$W_{cyc}$について、
                \begin{equation}
                    W_{cyc} \le 0
                \end{equation}
                が成り立つ。これを\textbf{ケルヴィンの定理}といい、これは\textbf{熱力学第二法則}と同値な表現である。
                この原理の主張は、\textbf{等温サイクルにおいて、系が外界に正の仕事をすることはない}ということである。
                これは、第二種永久機関が存在しないことを表している。


        \section{ヘルムホルツの自由エネルギー}

            \subsection*{最大仕事の原理}
                始状態を$(T; X)$、終状態を$(T; X')$とするとき、仕事の値は変化のさせ方に依存するので一意には定まらないが、仕事の最大値(\textbf{最大仕事})は一意に定まる。
                この最大仕事は、\textbf{等温準静操作で系が外界にする仕事}に等しい。\par
                $(T; X)\xrightarrow{i}(T; X')$から分かる通り、仕事の値は示量変数の変化量に依存する。
                そのため、仕事は示量的な性質である相加性をもつといえる。

            \subsection*{ヘルムホルツの自由エネルギーの定義}
                最大仕事が一意に定まることを利用して、基準となる状態を$X_0$として、以下のように\textbf{ヘルムホルツの自由エネルギー}を定義する。
                \begin{equation}
                    F(T; X) = W_{max}(T;X \xrightarrow{} X_0)
                \end{equation}
                これは、基準点$X_0$から$X$の状態に変化させるために必要な\textbf{最大仕事}を表している。
                もちろん、先ほど述べた最大仕事の相加性より、ヘルムホルツの自由エネルギーも同様の性質をもつことがわかる。\par
                任意の2点に対して、
                \begin{align}
                    W_{max}(T; X_1 \xrightarrow{} X_2) &= W_{max}(T; X_1 \xrightarrow{} X_0) + W_{max}(T; X_0 \xrightarrow{} X_2) \\
                    &= F(T;X_1 \xrightarrow{} X_0) + F(T;X_0 \xrightarrow{} X_2) \\
                    &= F(T; X_1) - F(T; X_2) \\
                \end{align}
                が成り立ち、\textbf{2点間の等温操作での系が外界にする仕事の最大値は、2点のヘルムホルツの自由エネルギーの差に等しい}といえる。

            \subsection*{圧力の定義}
                等温過程において、体積が微小変化する場合($V \rightarrow{} V + \Delta V$)を考える。このときの最大仕事は、
                \begin{align}
                    W_{max} &= F \Delta l + O(\Delta l^2) \\
                    &= P \Delta V + O(\Delta V^2)
                \end{align}
                等温操作における最大仕事がヘルムホルツの自由エネルギーに対応することから、
                \begin{equation}
                    P = -\frac{\partial}{\partial V} F(T; V, N)
                \end{equation}
                という関係式が成り立つことが分かる。


    \chapter{断熱操作}

        \section{断熱操作とは}
            \subsection*{断熱操作とは}
                断熱操作とは、系を断熱壁で覆われた環境下で変化させることである。
                矢印上部の$a$は断熱操作であることを表している。ちなみに、$aq$で断熱準静操作を表す。
                \begin{equation} \label{eq:3a}
                    (T; X)\xrightarrow{a}(T'; X')
                \end{equation}
                    
        \section{エネルギー}

            \subsection*{断熱仕事}
                断熱操作では、外部とのエネルギーのやり取りは力学的な仕事を介してのみ行われる。
                これから、力学でいう位置エネルギーと似たような性質を持っているといえる。\par
                (\ref{eq:3a})式のような断熱操作を考える。
                エネルギーのやり取りが仕事のみであるから、始状態と終状態が決定しているとき、その間での変化は全て系がする仕事によるものであるといえるので、その値は一意に決まる。
                これを\textbf{断熱仕事}と呼び、$W_{ad}((T; X)\xrightarrow{}(T'; X'))$と表す。
                この一意性などの性質はある種、等温操作における最大仕事に似ている。
                            
            \subsection*{内部エネルギーの定義}    
                等温操作の元で最大仕事に対してヘルムホルツの自由エネルギーを定義したように、断熱操作の元での断熱仕事に対して\textbf{内部エネルギー}を定義する。
                基準となる状態を$(T_0;X_0)$として、
                \begin{equation}
                    U(T; X) = W_{ad}((T;X) \xrightarrow{} (T_0;X_0))
                \end{equation}
                とする。これは、基準点$X_0$から$X$の状態に変化させるために必要な\textbf{断熱仕事}を表している。
                もちろんのこと、仕事に依存するような値なので相加性をもつ。\par
                任意の2点に対して、
                \begin{align}
                    W_{ad}((T_1;X_1) \xrightarrow{} (T_2;X_2)) &= W_{ad}((T_1;X_1) \xrightarrow{} (T_0;X_0)) + W_{ad}((T_0;X_0) \xrightarrow{} (T_2;X_2)) \\
                    &= U(T_1;X_1) - U(T_2;X_2) \\
                \end{align}
                が成り立ち、\textbf{2点間の断熱操作での系が外界にする仕事は、2点の内部エネルギーの差に等しい}といえる。
                この性質は\textbf{エネルギー保存則}や\textbf{熱力学第一法則}とも呼ばれる。


    \chapter{熱}
        \section{熱}
            \subsection*{熱の定義}
                断熱操作においては4章で扱った通り、系がされた仕事と内部エネルギーの変化量が一致した。
                つまり内部エネルギーの変化は一意に定まるような仕事によって表すことができるということである。\par
                しかし、等温操作においてはここまで単純な話にはならない。
                理由はシンプルで系と隣接した環境から何かしらのエネルギーを受け取ることができるからである。
                この得体の知れないエネルギーを\textbf{熱}と表現する。
                吸熱量を$Q$と表すことにすると、(\ref{eq:4a})式のような等温操作に対して、
                \begin{equation} \label{eq:4a}
                    (T; X) \xrightarrow{i} (T; X')
                \end{equation}
                \begin{equation} \label{eq:4b}
                    W = U(T; X) - U(T; X') + Q
                \end{equation}
                \begin{equation} \label{eq:4c}
                    U(T; X) - U(T; X') = -W + Q
                \end{equation}
                という関係式が成り立つ。特に(\ref{eq:4c})式は、内部エネルギーの変化は仕事と熱のやり取りに依存することを表している。\par
                ひとつ注意しておきたいのは、この熱$Q$という物理量は系の状態を表すようなものではない。
                つまり\textbf{非状態量}であり、エネルギーの伝搬に寄与するような量である。
                イメージとしては、エネルギーと仕事の関係でいう仕事の方である。

            \subsection*{最大級熱量}
                (\ref{eq:4c})式を見ると、等温操作において最大吸熱量を最大にするためには、仕事$W$を最大にすれば良い。
                これより、ヘルムホルツの自由エネルギーの定義を用いて、
                \begin{align}
                    Q_{max}(T; X \xrightarrow{} X') 
                    &= W_{max}(T; X \xrightarrow{} X') + U(T; X') - U(T; X) \\
                    &= F(T; X) - F(T; X') + U(T; X') - U(T; X)
                \end{align}
                と表すことができる。この表式はカルノーの定理で大変重要になってくる。

        \section{カルノーの定理}
            \subsection*{カルノーサイクル}
                以下のような、等温準静操作と断熱準静操作で構成された一周の操作を考える。
                \begin{equation}
                    (T'; X'_0) \xrightarrow{iq} (T'; X'_1) \xrightarrow{aq} (T; X_1) \xrightarrow{iq} (T; X_0) \xrightarrow{aq} (T'; X'_0)
                \end{equation}
                このサイクルのことを\textbf{カルノーサイクル}という。
                このサイクルにおいて、一周の間に系が外界にする仕事$W_{cyc}$は、計算の過程は省くが準静的操作であることを利用して、
                \begin{equation}
                    W_{cyc} = Q_{max}(T'; X'_0 \xrightarrow{} X'_1) - Q_{max}(T; X_0 \xrightarrow{} X_1)
                \end{equation}
                となる。等温準静操作の過程で吸収した熱を仕事として外界に放出していることが分かる。
                このカルノーサイクルについては、次の熱機関のセクションで再度詳しく扱う。

            \subsection*{カルノーの定理}
                カルノーサイクルにおいて、最大吸熱量の比
                \begin{equation}
                    f(T', T) = \frac{Q_{max}(T'; X'_0 \xrightarrow{} X'_1)}{Q_{max}(T; X_0 \xrightarrow{} X_1)}
                \end{equation}
                は、熱力学的な系の選択に依存せず、温度$T$、$T'$のみに依存する。
                これを\textbf{カルノーの定理}という。
                我々が絶対温度という尺度を持って温度を決定する場合、この定理はさらに、
                \begin{equation}
                    f(T', T) = \frac{T'}{T}
                \end{equation}
                と表される。カルノーサイクルが外部にする仕事は最大吸熱量の差であったが、その比は単純な温度のみに依存する関数となる。
                これは熱機関の効率について、非常に重要な結果である。

        \section{熱機関}    

            \subsection*{熱効率とその上限}
                \textbf{熱機関}とは、熱を仕事に変換するための装置のことである。
                一般的な熱機関でも、準静的ではないとはいえカルノーサイクルのように一周のサイクルを持っているを考えることができる。
                そのサイクルの中で、外部から受け取る熱量を$Q_H$、外部に放出する熱量を$Q_L$、外部にする仕事を$W$とすると、
                \begin{equation}
                    \epsilon = \frac{W}{Q_H} = 1 - \frac{Q_L}{Q_H}
                \end{equation}
                と\textbf{熱効率}を定義することができる。\par
                このサイクルがカルノーサイクルであれば、カルノーの定理により、
                \begin{equation}
                    \epsilon = 1 - \frac{T_L}{T_H}
                \end{equation}
                と温度の比で表現することができる。
    
    \chapter{エントロピー}
        \section{エントロピー}
            \subsection*{エントロピーの定義}
                
            \subsection*{エントロピーと断熱準静操作}
        \section{エントロピー原理}
            \subsection*{プランクの原理}
            \subsection*{エントロピー原理}
            \subsection*{エントロピー増大則}
        \section{エントロピーの熱的解釈}
            \subsection*{クラウジウス流エントロピーの解釈}
                (\ref{eq:5a})式のような等温準静操作を考える。
                \begin{equation} \label{eq:5a}
                    (T; X) \xrightarrow{iq} (T; X')
                \end{equation}
                この操作において、最大吸熱量$\Delta Q$ 、エントロピーの変化を$\Delta S$とすると、
                \begin{equation}
                    \Delta Q = Q_{max}(T; X \xrightarrow{} X')
                \end{equation}
                \begin{equation}
                    \Delta S = S(T; X') - S(T; X)
                \end{equation}
                と表される。このとき、この2つの量の間には
                \begin{equation}
                    \Delta S = \frac{\Delta Q}{T}
                \end{equation}
                のような関係式が成り立つ。
                この式を見ると、断熱準静操作においてエントロピーの変化が$0$であることが一目で分かる。\par
                それ以上にこの関係式の特筆すべき点は、熱という非状態量とエントロピーという状態量が等号で結ばれていることである。

    
    \chapter{ヘルムホルツの自由エネルギー}
    
    
    \chapter{ギブスの自由エネルギー}
        

\end{document}