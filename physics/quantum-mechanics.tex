\documentclass[a4paper]{jsreport}
\usepackage[utf8]{inputenc}
\usepackage{amsmath}
\usepackage{physics}

\title{量子力学 - Quantum Mechanics}

\begin{document}

    \maketitle

    \tableofcontents

    \chapter{オーバービュー}

        \section{量子とは}
            量子とは\textbf{波動性と粒子性の2つの性質をあわせ持つようなもの}のことである。
            具体的には、電子などのミクロな粒子のことである。
            この二面性を持つことから、どうしてもニュートン力学のような古典力学では対応しきらず、新しい力学が必要になった。

        \section{量子力学とは}
            量子の世界は非常に不思議な世界である。
            この不思議さは、よく二重スリットを用いた思考実験でよく語られる。
            ざっくり言うと、二重スリットの側で電子が通る様子を観察していると、明らかに粒が通っている。
            ただ、スクリーンには粒ではなく波が作るような干渉縞ができる。

            


    \chapter{波動関数と物理量}

        \section{波動関数}
            波動力学(シュレーディンガー形式)における波動関数を以下のように定義する。\par
            粒子の位置を測定した際に、微小体積$d \boldsymbol{r}$内に粒子が見出される確率は、
            \begin{equation}
                |\psi(\boldsymbol{r}, t)|^2 d\boldsymbol{r}
            \end{equation}
            に比例する。
            これは波動的な要請が非常に強い定義である。
            故に、実態が粒子であろうがなんであろうが、観測者である我々にとっては波動の性質の方が顕著に観測できるのだから、波動としての性質に重きを置いておこうという主張である。
            ちなみに、波動を表す関数の絶対値の2乗$|\psi(\boldsymbol{r}, t)|^2$が強度(=確率)を表現するというのは幾何光学の考え方と同様である。\par
            波動関数の2乗が確率を表現するということから、確率の合計が1になるように整えたい。
            この操作を\textbf{規格化}といい、
            \begin{equation}
                \int |\psi(\boldsymbol{r}, t)|^2 d\boldsymbol{r} = 1
            \end{equation}
            という条件を課すことで、確率の総和が1になる。
            この計算では、
            \begin{equation}
                \int |\psi|^2 d\boldsymbol{r} = \int \psi^* \psi d\boldsymbol{r}
            \end{equation}
            というように、波動関数が複素数であることを考慮して、複素共役を取る必要がある。
            この演算を\textbf{内積}とし、
            \begin{equation}
                (\psi, \psi) = \psi^* \psi
            \end{equation}
            というように表現する。
            名前だけ出しておくが、このように表現されるのは量子力学が\textbf{ヒルベルト空間}を舞台にした学問であるからであるといえる。

        \section{不確定性原理}
            干渉縞のような波動としての性質は、\textbf{大数の法則}に基づいて形成される。
            つまり、粒子ごとにスポットを当てて観測してみると、薄い干渉縞が現れるわけもなく、ただ一つの粒子がスクリーンに引っ付くだけである。
            波動関数の主張として、傾向はあれど確率的に動きが決まる。
            つまり、量子の性質として本質的に不確定な要素を含んでいる。
            ここでは、一次元での結果のみを記述しておくと、
            \begin{equation}
                \Delta x \Delta p \geq \frac{\hbar}{2}
            \end{equation}
            と表され、古典力学ではありえないような不確定性が生じることがわかる。

        \section{シュレーディンガー方程式}
            波動関数$\psi(\boldsymbol{r}, t)$は以下のようなシュレーディンガー方程式の解である。
            \begin{equation}
                i \hbar \frac{\partial \psi(\boldsymbol{r}, t)}{\partial t} = \left( -\frac{\hbar^2}{2m} \nabla^2 + V(\boldsymbol{r}, t)\right) \psi(\boldsymbol{r}, t)
            \end{equation}
            これの右辺について、
            \begin{equation}
                \hat{H} = -\frac{\hbar^2}{2m} \nabla^2 + V(\boldsymbol{r}, t)
            \end{equation}
            と定義したとき、この$\hat{H}$を\textbf{ハミルトニアン}という。
            この命名は、解析力学におけるハミルトニアンに由来しており、これはエネルギーの次元を持っている。
            ここで、古典力学との対応を考えたいのだが、ハミルトニアンを古典的に表現すると、エネルギーの次元を持つことから、
            \begin{equation}
                H = \frac{p^2}{2m} + V(x)
            \end{equation}
            と表現できる。この古典的な表式とハミルトニアン$\hat{H}$を比べると、運動量の対応関係として、
            \begin{equation}
                p \rightarrow -i \hbar \nabla
            \end{equation}
            であることがわかる。
            これを\textbf{運動量演算子}と呼び、演算子と値を区別するために、$\hat{p}$のように表現することにする。

        \section{固有値と期待値}
            物理量と固有値
            固有値=期待値
            エーレンフェストの定理

        \section{定常状態}
            定常状態のシュレーディンガー方程式

        \section{群速度と位相速度}
            分散

    \chapter{一粒子系}
        この章では、簡易的なモデルを通じて、量子の性質について考えていくことにする。
        ただ、計算過程を書いていく気力はないので、計算結果のみを記述していくことにする。
        \section{3次元自由粒子}
            基底状態や励起状態
            ゼロ点運動やゼロ点エネルギー
            縮退やエネルギー準位

        \section{球対称ポテンシャル}
            原子などのような球対称ポテンシャルをもつモデルを考える際には、デカルト座標系での表示よりも極座標形式の方が良い。
            そこで、シュレーディンガー方程式を極座標形式に書き換えると、
            \begin{equation}
                -\frac{\hbar^2}{2m} \left( \frac{1}{r^2} \frac{\partial}{\partial r} r^2 \frac{\partial}{\partial r} + \frac{1}{r^2 \sin \theta} \frac{\partial}{\partial \theta} \sin \theta \frac{\partial}{\partial \theta} + \frac{1}{r^2 \sin^2 \theta} \frac{\partial^2}{\partial \phi^2} \right) \varphi(r, \theta, \phi) + V(r) \varphi(r, \theta, \phi) = E \varphi(r, \theta, \phi)
            \end{equation}
            となる。
            ここで、波動関数$\varphi(r, \theta, \phi)$を
            \begin{equation}
                \varphi(r, \theta, \phi) = R(r) Y(\theta, \phi)
            \end{equation}
            と書き換えることにより、
            \begin{align}
                -\frac{\hbar^2}{2m} \left( \frac{1}{r^2} \frac{d}{dr} r^2 \frac{d}{dr} R(r) \right) + V(r) R(r) = E R(r) \\
                \left( \frac{1}{\sin \theta} \frac{\partial}{\partial \theta} \sin \theta \frac{\partial}{\partial \theta} + \frac{1}{\sin^2 \theta} \frac{\partial^2}{\partial\phi^2} \right) Y(\theta, \phi) = \lambda Y(\theta, \phi)
            \end{align}
            というように動径方向と角度方向の2つに分離して固有値方程式を考えることができる。\par
            先に角度方向について考える。
            結果だけ記述しておくと、固有値方程式の固有値と固有ベクトルはそれぞれ
            \begin{equation}
                \left( \frac{1}{\sin \theta} \frac{\partial}{\partial \theta} \sin \theta \frac{\partial}{\partial \theta} + \frac{1}{\sin^2 \theta} \frac{\partial^2}{\partial\phi^2} \right) Y_l^m(\theta, \phi) = l(l+1) Y_l^m(\theta, \phi)
            \end{equation}
            と表される。
            この固有ベクトルは\textbf{球面調和関数}と呼ばれている。
            さらにこの固有ベクトルは、角運動量に対しても固有ベクトルとなり、
            \begin{align}
                \boldsymbol{l}^2 Y_l^m(\theta, \phi) = l(l+1) \hbar^2 Y_l^m(\theta, \phi) \\
                l_z Y_l^m(\theta, \phi) = m \hbar Y_l^m(\theta, \phi)
            \end{align}
            という2つの固有値方程式が成り立つ。
            これらの式から読み取れることは、後述するので一旦置いておく。\par
            次に、動径成分について考える。
            こちらについては、どんな系かによって結果が大きく変わってくるので、定性的な話だけに留めておく。
            動径方向の固有値方程式の固有値はエネルギーを表す。
            つまり、中心からの距離に関して調べてあげることで、エネルギーを計測することができ、このエネルギーは$n$という量子数に依存する。
            ここまで出てきた量子数について、$n$は\textbf{主量子数}、$l$は\textbf{方位量子数}、$m$は\textbf{磁気量子数}という。
            主量子数$n$はエネルギーの値に直接的に関係しており、エネルギー準位を決定するような量子数であるといえる。
            方位量子数$l$は、角運動量の2乗$\boldsymbol{l}^2$の固有値として現れるので、角運動量の大きさを表現するような量子数であるといえる。
            磁気量子数$m$は、$l_z$の固有値として現れるため、$z$軸方向の角運動量を表現している量子数であるといえる。
            ここで重要なことは、$l_x$と$l_y$については確定値が得られないということである。
            $l_z$には固有値方程式があり、固有値が確定値として得られるが、$l_x$と$l_y$は球面調和関数を固有ベクトルとしていない。
            実際、
            \begin{align}
                l_+ &= l_x + i l_y \\
                l_- &= l_x - i l_y
            \end{align}
            というようにおくことで、
            \begin{align}
                l_+ Y_l^m(\theta, \phi) &= \hbar \sqrt{l(l+1) - m(m+1)} Y_l^{m+1}(\theta, \phi) \\
                l_- Y_l^m(\theta, \phi) &= \hbar \sqrt{l(l+1) - m(m-1)} Y_l^{m-1}(\theta, \phi)
            \end{align}
            という固有値方程式が成り立つ。
            これは、雰囲気として生成消滅演算子のような働きをするような演算子であると考えることができる。\par
            このように$z$方向のように、ある特定の方向の固有値は確定値として得られるが、他の2つの方向については確定値が得られない。
            これはある種の不確定性のようなものであり、\textbf{一つの方向に注目して角運動量を測定することはできるが、それ以外の方向に対しては確定値を得られない}ということを表している。
            これらの考え方は、電子が内在的に持つと考えられるスピンの角運動量を定義するときに活用される。

    \chapter{行列と状態ベクトル}

        \section{行列と量子力学}
            ここでは、量子力学を行列形式で表すこととする。
            この章以後は、行列とベクトルを用いて量子力学の議論を進めていくこととする。\par
            これまで扱ってきた関数はベクトルへ、演算子は行列として表現される。
            例えば、波動関数$\psi(\boldsymbol{r})$は、
            \begin{equation}
                \psi(\boldsymbol{r}) \rightarrow \ket{\psi}
            \end{equation}
            というように\textbf{ブラケット記法}で表現する。
            ちなみに、このベクトルを\textbf{ケットベクトル}といい、このベクトルの複素共役(双対ベクトル)は、
            \begin{equation}
                \ket{\psi}^\ast = \bra{\psi}
            \end{equation}
            と表現され、これを\textbf{ブラベクトル}という。
            これらの定義を用いると、内積は、
            \begin{equation}
                (\psi, \psi) = \braket{\psi}{\psi}
            \end{equation}
            と記述することになる。
            もちろん物理量$F$の期待値は
            \begin{equation}
                \ev{F} = \mel{\psi}{F}{\psi}
            \end{equation}
            と表現される。

        \section{基底変換}
            量子力学において、基底の変換には\textbf{ユニタリー行列}と呼ばれるものを使用する。
            ユニタリー行列というのは、
            \begin{equation}
                T^{-1} = T^\ast \quad  TT^\ast = T^\ast T = E
            \end{equation}
            を満たすような行列$T$のことである。
            状態ベクトルを$V$、何かしらの演算子を$A$をおくと、それぞれは
            \begin{align}
                V' &= TV \\
                A' &= TAT^{-1} = TAT^\ast
            \end{align}
            というように変換される。
            演算子は少し特徴的な変換が施されるが、これは固有値を保持するために、変換後の演算子に直接固有値を渡すのだと考えれば直感的にも捉えやすいと思う。
            固有値はその物理系の情報を持っているものであり、変換に際してそれが変更されてしまってはいけないからである。

        \section{交換関係}
            

        \section{ハイゼンベルグの運動方程式}

    \chapter{摂動論}
        
            
    \chapter{スピン}
        \section{スピン角運動量}
            電子はスピンによる(と考えると直感的かつ古典的な)角運動量とそれに伴う磁気モーメントを持っている。
            \textbf{スピン角運動量}を$s$と表記し、天下り的に$s=\pm \frac{1}{2}$であることとして、角運動量との対応関係を考えることで、固有状態を$\xi$とすると、
            \begin{equation}
                s^2 \xi = s(s+1)\hbar^2 \xi = \frac{3}{4}\hbar^2 \xi
            \end{equation}

        \section{スピン軌道相互作用}


        \section{角運動量の合成}


        \section{ゼーマン効果}
            

    \chapter{多粒子系}
    \chapter{数表示}
    \chapter{第二量子化}
    \chapter{相対論的量子力学}
        
\end{document}
