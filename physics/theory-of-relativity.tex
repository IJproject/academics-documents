\documentclass[a4paper]{jsreport}
\usepackage[utf8]{inputenc}
\usepackage{amsmath}

\title{相対性理論 - Theory of Relativity}

\begin{document}

    \maketitle

    \tableofcontents

    \chapter{オーバービュー}
        \section{相対性理論とは}
            ここは後々書くヲ。

    \chapter{ローレンツ変換}
        \section{特殊相対性原理}
            種々の物理現象に対して、次のような2つの原理を要請する。\par
            1つ目の要請は、\textbf{物理法則はすべての慣性系に対して同じ形で表される}という原理である。
            例として、電子誘導の法則について考える。
            円形の銅線と一本の棒磁石を用意し、この2つを近づけたり遠ざけたりする場合を考える。
            この時、銅線を動かしても棒磁石を動かしても、相対速度が同じであれば同様の電流が流れる。
            つまり、どちらの慣性系に対しても同じような定式が得られるはずである。
            もちろん、ここで考えている系は慣性系ではないのだが、例としてイメージしやすいので挙げたことだけは注意してほしい。\par
            2つ目の要請は、\textbf{真空中の光速は光源の運動状態に無関係である}という原理である。
            これはマクスウェル方程式において、どの慣性系に対しても電磁波の伝搬速度が不変であるという性質から自然と出てくる原理である。
            これはつまり、どの速度で進んでいる観測者から見ても、光の速度は変わらないということである。\par
            以上の2つの原理を合わせて、\textbf{特殊相対性原理}という。ちなみに、一般相対論は加速度系も含んだような理論になっており、今回考えているような慣性系に限定した理論ではない。
            そのため、特殊という接頭辞がついているかと思われる。
            
        \section{時空座標}
            特殊相対性原理のもとでは、座標だけではなく時間についても座標系に依存した量になる。
            時間が絶対的でなくなる理由は、光速が不変であるという原理を据えたからである。
            どこかしらの座標上の点を指定するためには、座標の情報と時間の情報の4つのパラメータが必要になる。
            それを$(x, y, z, t)$と表し、これを\textbf{時空座標}といい、時空座標上の点を\textbf{時空点}や\textbf{世界点}という。

        \section{ローレンツ変換}
            ここから本題のローレンツ変換に入ろうと思う。
            ローレンツ変換というのは、慣性系の間での座標変換である\textbf{ガリレイ変換}に、特殊相対性原理を適用した変換のことである。
            そのため、ガリレイ変換と大きく違うのは、\textbf{時間が絶対的ではない}ことで、言い換えると座標系の間で時間が共有されないことである。
            つまり、時空座標の定義の際に述べたとおり、座標系ごとに異なる時間軸をもつ。
            そのため、時空座標上の点を指定するために時間$t$というパラメータを追加したのである。\par
            実際に、座標系間の対応関係を求めてみる。
            静止した座標系$S$と、それに対して$x$方向に速度$V$で移動している座標系$S'$を考える。
            この時、$x$軸方向の変換式は、比例定数$\gamma(V)$を導入して、
            \begin{equation}
                x' = \gamma(V)(x - Vt)
            \end{equation}
            と表される。
            比例定数$\gamma(V)$は、座標系の移動速度に関連して座標軸が伸び縮みすることを反映させているためである。(ローレンツ収縮については後述)
            これに対して、$S'$からの視点で
            \begin{equation}
                x = \gamma(V)(x' + Vt')
            \end{equation}
            も成り立つ。
            この2つをの表式を用いて$t'$について解くと、
            \begin{equation}
                t' = \gamma(V)t - \frac{(\gamma(V))^2 - 1}{\gamma(V)V}x               
            \end{equation}
            と時間軸についての返還式を求めることができる。
            ここから先の議論については詳しくは触れないが、光速不変の原理に基づいて、球面波の方程式を考えることで$\gamma(V)$を求めることができて、
            \begin{equation}
                \gamma(V) = \frac{1}{\sqrt{1 - \frac{V^2}{c^2}}}
            \end{equation}
            と表される。
            ちなみにこの$\gamma$は\textbf{ローレンツ因子}と呼ばれている。
            これらを踏まえると、変換公式は
            \begin{align}
                x' &= \gamma(V)(x - Vt) = \frac{x - Vt}{\sqrt{1 - \frac{V^2}{c^2}}} \\
                y' &= y \\
                z' &= z \\
                ct' & = \gamma(V) \left(t - V\frac{x}{c} \right)= \frac{t - V\frac{x}{c}}{\sqrt{1 - \frac{V^2}{c^2}}}
            \end{align}
            というように表され、これを\textbf{ローレンツ変換}という。
            $ct'$というように表現している理由は、次元を他の3つの量に合わせ、この後定義する距離の概念にそのまま使用するためである。
            ガリレイ変換では、光速は限りなく大きいという仮定の元での議論なので、$c \to \infty$とすると、これはガリレイ変換に一致する。

        \section{世界距離}
            特殊相対性原理において、光の球面波の方程式は例外なく満たされる。
            \begin{equation}
                x^2 + y^2 + z^2 -c^2t^2 = 0
            \end{equation}
            ここで、時空座標上の2点$(x_1, y_1, z_1, t_1)$、$(x_2, y_2, z_2, t_2)$における距離を
            \begin{equation}
                {s_{12}}^2 = (x_2 - x_1)^2 + (y_2 - y_1)^2 + (z_2 - z_1)^2 - c^2(t_2 - t_1)^2
            \end{equation}
            と定義する。
            この${s_{12}}^2$を\textbf{世界距離}という。
            世界距離は2つの世界点を結んだ曲線(\textbf{世界線})が、直線になった時の長さのことである。
            さらに、このように距離を定義したような空間を\textbf{ミンコフスキー空間}という。\par
            ここで一つ重要な結論として、\textbf{世界距離${s_{12}}^2$はローレンツ変換に対して不変}という性質を持つことを挙げる。
            これは光の球面波の方程式がどの慣性系についても成り立つことから、当たり前と言ってしまうこともできそうである。\par
            さらに、球面波の方程式を見れば分かるとおり、光速で考える場合であれば、${s_{12}}^2$の値は$0$になる。
            逆に言えば、光でなければそうならず、${s_{12}}^2$は正の数や負の数を取りうる。
            正の数を取るときは、座標変化の影響が大きい時で、負の数を取るときは、時間変化の影響が大きい時であるということができる。\par
            最後に補足として、全微分の表式を記述しておく。
            \begin{equation}
                ds^2 = dx^2 + dy^2 + dz^2 - c^2dt^2
            \end{equation}

        \section{固有時間}
            時空座標を考えることのできる座標系は、ミンコフスキー空間であり、この空間は世界距離という尺度を持った空間として定義されている。
            そこで、この空間の時間軸を世界距離を使って、
            \begin{equation}
                {\tau_{12}}^2 = - \frac{{s_{12}}^2}{c^2} 
            \end{equation}
            と定義する。
            この量について、例えば座標系に対して静止している時空点では、
            \begin{equation}
                {\tau_{12}}^2 = (t_2 - t_1)^2
            \end{equation}
            であることは、実際に計算すれば分かる。
            この値は、座標系に付随する時間軸を表しており、\textbf{固有時間}という。
            この${\tau_{12}}^2$について少し掘り下げてみると、この値は\textbf{その座標系の2点において、どれだけ時間が経過するかを表現している}。
            光速で進む場合、${s_{12}}^2 = 0$となることから、${\tau_{12}}^2 = 0$となり、時間が進まないことが分かる。
            巷で、光速で進む場合は時間の進みが遅くなるというセリフが独り歩きしているが、それがこのように時間軸を定義することで表式として得られる。\par
            最後に補足として、全微分の表式を記述しておく。
            \begin{equation}
                d\tau = dt \sqrt{1 - \frac{V^2}{c^2}} = \gamma dt
            \end{equation}
            ここでまたローレンツ因子が現れたことには、どこか美しさのようなものを感じるだろう。

        \section{時間の遅れ}

        \section{ローレンツ収縮}  
        
    \chapter{4元ベクトルによる定式化}
        \section{2次元時空}
            4元ベクトルを定義する前に2次元時空における定式を見てみる。
            4元ベクトルに拡張するにしても、キーになるのは$ct$と$x$のみなので、ここでの議論が比較的本質的なものになると思っている。\par
            さて、まずは表記法を以下のようにする。
            \begin{equation}
                x^0 = ct \quad x^1 = x 
            \end{equation}
            すると、ローレンツ変換の表式より、
            \begin{align}
                {x^0}' &= \alpha^0_0 x^0 + \alpha^0_1 x^1 \\
                {x^1}' &= \alpha^1_0 x^0 + \alpha^1_1 x^1
            \end{align}
            というように係数を設定して表すことができる。
            ちなみに、
            \begin{align}
                \alpha^0_0 &=  \gamma \\
                \alpha^0_1 &= - \frac{V}{c} \gamma \\
                \alpha^1_0 &= - \frac{V}{c} \gamma \\
                \alpha^1_1 &= \gamma
            \end{align}
            である。
            ローレンツ変換において、世界距離$s^2$は不変であることから、
            \begin{equation}
                s^2 = - (x^0)^2 + (x^1)^2 = - ({x^0}')^2 + ({x^1}')^2
            \end{equation}
            となることを利用すると、以下のような関係式を得ることができる。
            \begin{equation}
                (\alpha^0_0)^2 - (\alpha^0_1)^2 = (\alpha^1_0)^2 - (\alpha^1_1)^2 = -1
            \end{equation}
            \begin{equation}
                \alpha^0_0 \alpha^1_0 - \alpha^0_1 \alpha^1_1 = 0
            \end{equation}
            一つ目の式は、双曲線関数で見られる表式と同じ形をしている。
            ここでは詳しくは言及しないが、この性質から\textbf{世界距離を導出する際に$c^2t^2$の項が負になっていること}や\textbf{ミンコフスキー空間が複素空間的な性質を持っている}ことを示すことができる。\par
            さて、ここまでは2章でやったローレンツ変換を2次元時空に拡張しただけだったが、ここから行列的な表現を導入する。
            まず、以下のように変換行列と、ベクトルを定義する。
            \begin{equation}
                A = \begin{pmatrix}
                    \alpha^0_0 & \alpha^0_1 \\
                    \alpha^1_0 & \alpha^1_1
                \end{pmatrix} \quad
                \mathbf{x} = (x^\mu) = \begin{pmatrix}
                    x^0 \\
                    x^1
                \end{pmatrix} \quad
                \mathbf{x'} = ({x^\mu}') = \begin{pmatrix}
                    {x^0}' \\
                    {x^1}'
                \end{pmatrix}
            \end{equation}
            すると、ローレンツ変換は以下のように表される。
            \begin{equation} \label{eq:3k}
                \mathbf{x'} = A \mathbf{x}
            \end{equation}
            ここでは結果のみを記述しておくが、逆変換の行列$A^{-1}$は、
            \begin{equation} 
                A^{-1} = \begin{pmatrix}
                    \alpha^1_1 & -\alpha^0_1 \\
                    -\alpha^1_0 & \alpha^0_0
                \end{pmatrix}
            \end{equation}
            と表される。
            こちらも結果のみを記述しておくが、基本テンソル$H$というものを導入することで、
            \begin{equation}
                H =  \begin{pmatrix}
                    -1 & 0 \\
                    0 & 1
                \end{pmatrix} \quad
                s^2 = ^t\!\mathbf{x} H \mathbf{x} = ^t\!\mathbf{x'} H \mathbf{x'}
            \end{equation}
            とローレンツ変換における不変量である世界距離$s^2$を表現することができる。
            このテンソルについては後に詳しく触れるが、反変ベクトルと共変ベクトル間の変換などに利用される。
            直感として、$-1$が入っている1行1列の部分は、時間の部分が複素数的な性質を示すことを暗に表現していると捉えることができる。
            
        \section{4元ベクトル}
            先ほどまでは2次元時空について考えていたが、それを4次元に拡張する。
            \begin{equation}
                (x^\mu) = (x^0, x^1, x^2, x^3)
            \end{equation}
            のように定義したものを\textbf{4元位置ベクトル}といい、これを用いて、
            \begin{equation}
                (u^\mu) = (u^0, u^1, u^2, u^3) = \left( \frac{dx^0}{d\tau}, \frac{dx^1}{d\tau}, \frac{dx^2}{d\tau}, \frac{dx^3}{d\tau} \right)
            \end{equation}
            を\textbf{4元速度ベクトル}という。
            微分自体は、固有時間$\tau$で行われていることに注意してほしい。
            4元ベクトル自体は、一つの慣性系上で定義されるものなので、微分をする際はその慣性系に付随する固有時間で微分しないといけないからである。

        \section{共変ベクトルと反変ベクトル}
            座標系の変換は、(\ref{eq:3k})式で表された。
            このようなベクトルを\textbf{反変ベクトル}という。
            この言葉の意味は後で説明する。
            これに対して、基底ベクトルの変換では、
            \begin{equation}
                \mathbf{e'} = A^{-1} \mathbf{e}
            \end{equation}
            のように、変換をする際に逆行列を用いる。
            このようなベクトルを\textbf{共変ベクトル}という。
            つまり、ローレンツ変換をする際に、\textbf{基底の変換と同じように変換するベクトルを共変ベクトル}といい、\textbf{基底の変換と逆の変換をするベクトルを反変ベクトル}という。
            そして、反変ベクトル$(a^{\mu})$と共変ベクトル$(a_\mu)$はそれぞれ以下のような添え字を使って表現する。
            \begin{equation}
                (a^{\mu}) = (a^0, a^1, a^2, a^3) \quad (a_{\mu}) = (a_0, a_1, a_2, a_3)
            \end{equation}
            この2つのベクトルの間には、
            \begin{equation}
                (a_{\mu}) =  (a_0, a_1, a_2, a_3) = (-a^0, a^1, a^2, a^3) 
            \end{equation}
            のような関係があり、第0成分の符号だけが違っている。
            2次元時空の章で定義した基本ベクトル$H$を4次元に拡張すると、
            \begin{equation}
                H = \begin{pmatrix}
                    -1 & 0 & 0 & 0 \\
                    0 & 1 & 0 & 0 \\
                    0 & 0 & 1 & 0 \\
                    0 & 0 & 0 & 1
                \end{pmatrix}
            \end{equation}
            と表されることから、
            \begin{equation}
                a_{\mu} = H a^{\mu}
            \end{equation}
            という関係が成り立つ。

        \section{ローレンツ変換の4元表式}
            ここまでしてきた議論を元に、ローレンツ変換の表式を4元ベクトルを用いて表現する。
            先に断っておくが、ここから先は\textbf{アインシュタインの縮約記法}に基づいた記法でローレンツ変換の表式を

    \chapter{相対論的力学}
    \chapter{相対論的電磁気学}
    \chapter{一般相対論達観}


\end{document}