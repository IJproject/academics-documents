\documentclass[a4paper]{jsreport}
\usepackage[utf8]{inputenc}
\usepackage{amsmath}
\usepackage{physics}

\title{統計力学 - StatiStical Mechanics}

\begin{document}

    \maketitle

    \tableofcontents

    \chapter{オーバービュー}

        \section{統計力学とは}
            気が向いたら書くヲ

    \chapter{確率論と測定}
            
            統計力学で扱う系は、多数の粒子からなる系である。
            その系の各粒子に対して、それぞれの位置や運動量などの物理量を考えることは不可能に等しい。
            そこで、適切な確率モデルを用いて記述することにより、物理学的な現象を記述していく。
            それに必要な理論の準備をこの章で行う。

        \section{確率と期待値}
            確率の満たすべき性質は、
            \begin{align}
                p_i \ge 0 \\
                \sum_{i=1}^{\infty} p_i = 1
            \end{align}
            のように、非負性と規格化条件を満たすことである。
            もちろん、統計力学で使用する確率モデルもこれらを満たすように定義されなければならない。\par
            次に、事象$A$に対して、
            \begin{equation}
                \chi_i[A] = \begin{cases}
                    1 & \text{if $A$ occurs} \\
                    0 & \text{otherwise}
                \end{cases}
            \end{equation}
            というように状態$i$で$A$が起こるかどうかを表現できるような関数を用意することで、
            \begin{equation}
                Prob_p[A] = \sum_{i=1}^{N} p_i \chi_i[A]
            \end{equation}
            というように事象$A$が生じる確率を定義することができる。\par
            次に\textbf{期待値}を定義する。
            期待値は、
            \begin{equation}
                \ev{\hat{f}} = \sum_{i=1}^{N} f_i p_i 
            \end{equation}
            と定義され、ある物理量$\hat{f}$について測定したときに得られる測定値(確定値)のことである。
            例えば、ハミルトニアンであれば期待値はエネルギーとなる。

        \section{物理量のゆらぎ}
            ある物理量$\hat{f}$の測定をした際、その期待値だけを測定値として得られるわけではない。
            多少の振れ幅があり、その振れ幅を\textbf{ゆらぎ}というのだが、そのゆらぎが期待値に収束するといったことである。
            ゆらぎ$\sigma[\hat{f}]$は、
            \begin{equation}
                \sigma[\hat{f}] = \sqrt{\ev{\hat{f}^2} - \ev{\hat{f}}^2}
            \end{equation}
            と計算することができる。
            この値が大きいほど、得られる測定値のばらつきが大きいということになる。\par
            ここで、重要なゆらぎに関する不等式である、\textbf{チェビシェフの不等式}について触れておく。
            これは、任意の正の実数$\epsilon$に対して、
            \begin{equation}
                Prob_p \left[\abs{\hat{f} - \ev{\hat{f}} \ge \epsilon} \right] \le \frac{\sigma_p[\hat{f}]^2}{\epsilon^2}
            \end{equation}
            という関係が成り立つ。
            この$\epsilon$は測定精度を表していて、$\sigma_p[\hat{f}] \ll \epsilon$が成り立つ時、測定された値が真の値に非常に近いということを意味する。
            確率モデルに基づいた理論を構築する関係上、こういった測定誤差などについて考慮することは非常に重要である。
            さらに、この不等式を使用することで\textbf{大数の法則}を導くことができる。

    \chapter{量子論}
        \section{エネルギー固有値}
            量子論において、特定のエネルギー状態が確定しているとき、それを\textbf{エネルギー固有状態}といい、それに対応するエネルギーを\textbf{エネルギー固有値}という。
            統計力学では、このエネルギー固有値は非常に重要で、各固有状態に対してラベルを振り、それぞれのエネルギー固有値を使用した確率モデルを構築していく(カノニカル分布など)。\par
            エネルギー固有値は、シュレーディンガー方程式を解くことで求めることができる。
            一次元自由粒子モデルの場合、固有状態を$n=1, 2...$、固有値を$E_n=E_0 n^2$という解が得られる。
            この結果を3次元自由粒子N個の系に拡張して、

        \section{状態数}


    \chapter{カノニカル分布}
        \section{マクロな系との繋がり}
            マクロな体系を持つ熱力学との整合性を保つように、ミクロな体系を持つ統計力学は確率モデルを導入して理論を構築する。
            つまり、熱力学での制約は統計力学の中でも満たされなければならない。\par
            制約の一つ目は、\textbf{平衡状態$(U; V, N)$について、マクロな性質はただ一つに定まる}ということである。
            統計力学では、エネルギーが$U$の状態全てを含め、\textbf{許されるエネルギー状態}と表現する。
            この許されるエネルギー状態のうち、共通して見られる性質(\textbf{典型的な性質})がマクロな性質として現れると考える。\par
            制約の二つ目は、\textbf{平衡状態への緩和}である。
            これは、典型的な状態に収束するということから満たすことができる。\par
            統計力学における確率モデルを導入する際、\textbf{等重率の原理}という制約を与える。
            この原理は、許される量子状態それぞれに同じ重み(同様に確からしいという考え方と似ている)を与えるというものである。
            このような考え方ができるのは、許される量子状態それぞれを区別することが難しいという、測定の観点から見ると自然な考え方である。\par

        \section{ミクロカノニカル分布}
        \section{カノニカル分布}
            ミクロカノニカル分布よりも汎用性が高い確率分布である\textbf{カノニカル分布}を導入する。
            ミクロカノニカル分布を導出した時と同様に、状態数から議論を進めていくことで、
            \begin{equation}
                p_i = \frac{e^{-\beta E_i}}{\sum\limits_{i} e^{-\beta E_i}}
            \end{equation}
            と確率を表現することができる。
            ちなみに、
            \begin{equation}
                Z(\beta) = \sum_{i} e^{-\beta E_i}
            \end{equation}
            は分配関数と呼ばれている。
            粒子を区別するべきではないとき、分配関数は
            \begin{equation}
                Z(\beta) = \frac{1}{N!} \sum_{i} e^{-\beta E_i}
            \end{equation}
            と表される。
            ここまでの議論を踏まえて、物理量$\hat{f}$の期待値を計算すると、
            \begin{equation}
                \ev{\hat{f}} = \sum_{i}f_ip_i = \frac{1}{Z(\beta)} \sum_{i}f_i e^{-\beta E_i}
            \end{equation}
            となる。

        \section{エネルギーの期待値とゆらぎ}
            得られた期待値の表式を用いて、ハミルトニアン$\hat{H}$の固有値(エネルギー固有値)を求める。
            計算の過程は省略するが、
            \begin{equation}
                \ev{\hat{H}} = -\frac{\partial}{\partial \beta} \log{Z(\beta)}
            \end{equation}
            と表される。
            さらにエネルギー固有値のゆらぎについては、固有値を求めたときと同様の計算をすることで、
            \begin{equation}
                \sigma[\hat{H}] = \sqrt{\frac{\partial^2}{\partial \beta ^2} \log{Z(\beta)}}
            \end{equation}
            さらに、熱容量については、
            \begin{equation}
                C(T) = \frac{d}{dT} \ev{\hat{H}} = \frac{1}{kT^2}\frac{d^2}{d \beta ^2}\log{Z(\beta)} = \frac{1}{kT^2}(\sigma[\hat{H}])^2
            \end{equation}
            と表すことができる。

        \section{熱力学との関係性}
            ギブス・ヘルムホルツの関係式は、ヘルムホルツの自由エネルギー$F$を用いて、
            \begin{equation}
                U(T; V, N) = \frac{\partial}{\partial\beta}{\beta F(\beta, V, N)}
            \end{equation}
            と表現される。
            これから、$F$は測定精度を表していて
            \begin{equation} \label{eq:4a}
                F(\beta, V, N) = - \frac{1}{\beta} \log{Z(\beta)}
            \end{equation}
            と表される。
            このヘルムホルツの自由エネルギーの表式を利用することで
            \begin{equation}
                S = \frac{U - F}{T}
            \end{equation}
            という関係式からエントロピー$S$を求めることができる。
            このエントロピーという量は、統計力学において重要な意味をもつ量である。

    % \chapter{カノニカル分布の応用例}
    %     \section{磁性}
    %     \section{古典的描像}


    % \chapter{格子振動}

    % \chapter{電磁場}

    \chapter{グランドカノニカル分布}
        \section{グランドカノニカル分布}
            カノニカル分布は、各エネルギー固有状態に通し番号を振り、それぞれに等重率の原理を適用するような確率モデルであった。
            グランドカノニカル分布はそれに加えて、粒子数の変化も考慮した確率モデルになっている。
            つまり、扱う各平衡状態は$(N, i)$で指定することになる。
            ここからは便宜上、粒子を区別することが原理的に不可能なことから、平衡状態を$(N_i, E_i)$と表現することにする。\par
            まず、グランドカノニカル分布における平衡状態$(N_i, E_i)$の起こる確率は、カノニカル分布の導出方法と同様にすることで、
            \begin{equation}
                p_i = \frac{1}{\Xi(\beta, \mu)} \exp(-\beta E_i + \beta \mu N_i)
            \end{equation}
            と表される。
            分母に現れた$\Xi_V(\beta, \mu)$は\textbf{大分配関数}と呼ばれ、
            \begin{equation}
                \Xi(\beta, \mu) = \sum_{i=1}^{\infty}\exp(-\beta E_i + \beta \mu N_i) 
            \end{equation}
            と表される。
            ここで出てきた$\mu$は\textbf{化学ポテンシャル}と呼ばれる量で、\textbf{系の粒子数の変化に対して、系のエネルギーがどのように変化するのか}を表す値である。
            もちろん、上記の確率の表式を用いることで、期待値と測定値の揺らぎについての式を導出することが可能であるが、自明すぎるのでここでは省略する。

        \section{グランドカノニカル分布の諸性質}
            計算過程については省略するが、大分配関数を偏微分することで、
            \begin{align}
                \ev{N} &= \frac{1}{\beta} \frac{\partial}{\partial\mu} \log \Xi(\beta, \mu) \\
                \ev{H} &= - \frac{\partial}{\partial \beta} \log \Xi(\beta, \mu) + \frac{\mu}{\beta}\frac{\partial}{\partial \mu} \log \Xi(\beta, \mu)
            \end{align}
            と期待値を計算することができる。
            カノニカル分布では、(\ref{eq:4a})式のような関係式で、分配関数と自由エネルギーを結びつけていた。
            これをグランドカノニカル分布でも適用すると、
            \begin{equation}
                J(V, \beta, \mu) = - \frac{1}{\beta} \log \Xi(\beta, \mu)
            \end{equation}
            と表現することになり、この$J(V, \beta, \mu)$を\textbf{グランドポテンシャル}という。
            このポテンシャルは、表式を見るとわかるように、ヘルムホルツの自由エネルギーに粒子数の変化を考慮させたような量である。

    \chapter{量子理想気体}
        \section{多粒子系の量子論}
        \section{状態の数}
        \section{平衡状態の記述(グランドカノニカル分布)}
        \section{理想フェルミ気体}
        \section{理想ボース気体}
        
\end{document}
