\documentclass[a4paper]{jsreport}
\usepackage[utf8]{inputenc}
\usepackage{amsmath}

\title{解析力学 - Analytical Mechanics}

\begin{document}

    \maketitle

    \tableofcontents

    \chapter{オーバービュー}
        \section{解析力学とは}
            ここは後々書くヲ(ニュートン力学との違いを中心に)

    \chapter{最小作用の原理とラグランジュ形式}
        \section{一般化座標と一般化運動量}
            解析力学では、\textbf{一般化座標$q_i(t)$}と\textbf{一般化運動量$p_i(t)$}の2つをもとに理論が形成されている。
            この一般化というのは、任意性を持つというニュアンスの接頭辞である。
            つまり、デカルト座標(xyz座標)のようなものに限定されず、任意の座標系について統一的に考えることができるということである。\par
            この後に\textbf{オイラー=ラグランジュ方程式}という解析力学における基礎方程式を導入するが、その方程式では一般化座標と一般化運動量を用いて運動方程式を記述することになる。
            これにより、\textbf{どのような座標系を選択しても、同一の形の方程式で計算をすることができる}という強みを持つ。

        \section{最小作用の原理とラグランジアン}
            $N$個の独立な力学変数$q_i(t)(i=1,2, ..., N)$で記述される$N$自由度系を考える。
            この時、以下のように\textbf{ラグランジアン}という量を定義する。
            \begin{equation}
                L = L(q(t), \dot{q}(t), t)
            \end{equation}
            \begin{align}
                q(t) &= (q_1(t), q_2(t), ..., q_N(t)) \\
                \dot{q}(t) &= (\dot{q}_1(t), \dot{q}_2(t), ..., \dot{q}_N(t))
            \end{align}
            ちなみに、このラグランジアンは直接的に$t$という引数を持つことから、\textbf{陽の時間依存性}を持つという。
            陽な時間依存性を持たなければ、一般化座標と一般化運動量のみでラグランジアンが表せるということになる。\par
            さて、ここで以下のような\textbf{作用$S$}という量を導入する。
            \begin{equation}
                S[q] = \int_{t_1}^{t_2} L(q(t), \dot{q}(t), t) dt
            \end{equation}
            この時、系の運動は$S[q]$が最小になる(\textbf{停留}する)ように決定される。
            これを\textbf{最小作用の原理}という。
            つまり、どのように運動をするのか(どのような経路を辿って運動するのか)は、この作用$S$によって決まる。

        \section{変分法とオイラー=ラグランジュ方程式}
            では、作用$S$が最小になるような経路を求めるためにはどうすればよいか。\par
            まず、$q(t)$から微少量ずれた経路$q(t) + \delta q(t)$を考える。この時の作用の変化量$\delta S[q]$は、$\delta$の2次以上の項を無視し、テイラー展開などを行うことで、
            \begin{align}
                \delta S[q] &= S[q + \delta q] - S[q] \\
                &= \int_{t_1}^{t_2} \left( L(q + \delta q, \dot{q} + \delta \dot{q}, t) - L(q, \dot{q}, t) \right) dt \\
                &= \int_{t_1}^{t_2} dt \left( \frac{\partial L}{\partial q} - \frac{d}{dt} \frac{\partial L}{\partial \dot{q}} \right) \delta q
            \end{align}
            と計算できる。
            行間が大分省略されているが、ご愛嬌ということで。
            ここで、$\delta S[q]$は極値付近(今回は最小付近)の時、変化が緩やかになることから、$\delta S[q] = 0$となると考えるのが自然である。
            つまり、最小作用の原理を満たすような方程式として、
            \begin{equation}
                \frac{\partial L}{\partial q} - \frac{d}{dt} \frac{\partial L}{\partial \dot{q}} = 0
            \end{equation}
            が得られ、これを\textbf{オイラー=ラグランジュ方程式}という。
            さらに、このように微小変化に対する応答(\textbf{変分$\delta S[q]$})が$0$になるような条件で議論を進める手法を\textbf{変分法}という\par
            アインシュタインの縮約ルールに基づいた表記としては、
            \begin{equation}
                \frac{\partial L}{\partial q_i} - \frac{d}{dt} \frac{\partial L}{\partial \dot{q_i}} = 0
            \end{equation}
            となる。

        \section{ラグランジアンの不定性}
            任意関数$G(q(t), t)$について、
            \begin{equation}
                L'(q(t), \dot{q}(t), t) = L(q(t), \dot{q}(t), t) + \frac{d}{dt} G(q(t), t)
            \end{equation}
            というラグランジアン$L'$を考える。
            この$L'$はオイラー=ラグランジュ方程式において$L$と全く同じ解を与える。
            つまり、$L$と$L'$は等価であるといえる。
            これは変分法により、2つのラグランジアンの変分を比較することで示すことができる。
            この性質がラグランジアンの適用範囲を広げることになる。

        \section{ラグランジュの未定乗数法}
            いつか書く

    \chapter{対称性と保存則}
        \section{時間並進の対称性}
            時間並進とは、時間を定数分だけずらすような操作のことである。
            そこで、
            \begin{equation}
                t \to t + t_0
            \end{equation}
            のように変更した場合に、ラグランジアンがどのように応答するかを考える。
            この時、ラグランジアンが陽の時間依存性を持たない場合、$q(t+t_0)$も元のオイラー=ラグランジュ方程式の解になる。
            つまり、どちらの運動も同じラグランジアンで表すことができるということである。\par
            逆に、ラグランジアンが陽の依存性を持つ(=$t$に直接的に依存する)場合、受ける力などが変わってくるため、運動の仕方に違いが生じることは明らかである。
            そのため、同じラグランジアンで表すことができないことは直感的にも理解できる。

        \section{空間並進の対称性}
            空間並進とは、空間座標を一様にずらすような操作のことである。
            「一様に」というのは、座標全体を一定分だけという意味である。
            そこで、
            \begin{align}
                q(t) &\to q(t) + q_0 \\
                \dot{q}(t) &\to \frac{d}{dt} ({q(t)} + q_0) = \dot{q}(t)
            \end{align}
            のように変更した場合に、ラグランジアンがどのように応答するかを考える。
            この時、ざっくりとして表現にはなるが、空間に一様性があれば$q(t) + q_0$も元のオイラー=ラグランジュ方程式の解になる。
            ここでいう空間の一様性というのは、\textbf{座標に依存するような場に関するパラメータが含まれていない}という意味である。
            もちろん場に依存するするようなパラメータが存在するのであれば、働く力などが変わってくることは容易に想像することができる。

        \section{空間回転の対称性}
            空間回転とは、空間座標を原点を中心として回転させるような操作のことである。
            空間回転については、行列式の値が$1$になる回転行列$R$を左から作用させて、
            \begin{equation}
                q(t) \to Rq(t)
            \end{equation}
            となり、これによりラグランジアンがどのように応答するのかを考える。
            この際、\textbf{方向に依存した変化がなければ、空間回転に対して対称性を持つ}ことが分かる。
            例えば、$3$次元空間で$x$軸方向のみに力が働いているような系の場合、空間回転させてしまうと系に対する力の働く向きが変わってしまうため、空間回転に対しては対称ではない。
            返って、中心力による束縛のような系では、空間回転しても分からないため、これは空間回転に対して対称であるといえる。

        \section{ガリレイ不変性}
        \section{ネーターの定理}
            先ほどまでの対称性についての議論を踏まえて、保存則について考える。
            一つの重要な定理として、\textbf{ラグランジアンが対称性を持っている(=不変である)場合、それに対応した保存量が存在する}。
            これを\textbf{ネーターの定理}という。
            つまり、時間並進の対称性を持つ場合、それに対応して何かしらの保存量が存在する。
            これが他の対称性についても成り立つということである。
            対称性の議論から保存則が導出できるということである。\par
            ではこれを数式で追ってみたい。
            保存量$Q$は陽な時間依存性を持たないことから以下のように表される。
            \begin{equation}
                \frac{d}{dt}Q(q, \dot{q}) = 0
            \end{equation}
            ここで、対称性の議論の部分でも行ったような微小変換を、より一般化した場合で考えたい。
            微小変換を表現する量を$F_i^A(q, \dot{q})$として、
            \begin{align}
                q_i(t) &\to q_i(t) + F_i^A(q, \dot{q}) \epsilon_A \\
                \dot{q_i}(t) &\to \dot{q_i}(t) + \frac{d}{dt} F_i^A(q, \dot{q}) \epsilon_A
            \end{align}
            これに対してラグランジアンがどのように応答するのかを考えるために、$\delta L$を計算すると、
            \begin{align}
                \delta L &= \frac{d}{dt} \frac{\partial L(q, \dot{q}, t)}{\partial \dot{q}} F_i^A \\
                &= \frac{d}{dt} Y^A(q, \dot{q}, t) \epsilon_A \\
                &= 0
            \end{align}
            ここから保存量$Q^A$は、
            \begin{equation} \label{eq:3a}
                Q^A = \frac{\partial L(q, \dot{q}, t)}{\partial \dot{q}} F_i^A(q, \dot{q}) - Y^A(q, \dot{q}, t)
            \end{equation}
            で与えられる。
            これにより、\textbf{どのように微小変換させるのか}を表す$F_i^A$と\textbf{ラグランジアンの変分}を表す$\delta L$を求めれば、保存量を導出することができることがわかる。

        \section{エネルギー保存則}
            3.1節で考えた、時間並進の対称性について、ネーターの定理を適用する。
            \begin{align}
                q_i(t) &\to q_i(t + \epsilon_0) = q_i(t) + \dot{q_i}(t)\epsilon_0 \\
                \dot{q_i}(t) &\to \dot{q_i}(t + \epsilon_0) = \dot{q_i}(t) + \ddot{q_i}(t)\epsilon_0
            \end{align}
            まず、ラグランジアンの変分は、全微分を計算することで、
            \begin{equation}
                \delta L = \frac{d}{dt} L(q, \dot{q})\epsilon_0
            \end{equation}
            ネーターの定理(\ref{eq:3a})式との対応関係を考えると、
            \begin{equation}
                E = \frac{\partial L(q, \dot{q})}{\partial \dot{q_i}} \dot{q_i} - L(q, \dot{q})
            \end{equation}
            という保存量$E$を得ることができ、これは一般に\textbf{エネルギー}と呼ばれるものである。
            つまり、\textbf{時間並進について対称(ラグランジアンが不変)の場合、エネルギー保存則が成り立つ}ということである。
            後に出てくるが、これはハミルトニアンの定義式と全く同じ式をしている。


        \section{運動量保存則}
        \section{角運動量保存則}

    \chapter{ハミルトン形式}
        \section{ハミルトニアン}
        ラグランジアンのルジャンドル変換で得られる
        \section{ハミルトンの運動方程式}
        最小作用の原理からも導出可能
        \section{ハミルトニアンの時間微分}
        \section{位相空間軌跡}
        \section{ポアソンブラケット}


    \chapter{正準変換}
        \section{ルジャンドル変換}
        \section{正準変換}
        \section{母関数$F(q, Q, t)$における正準変換}
        \section{母関数$\Phi(q, P, t)$における正準変換}
        \section{母関数$\Psi(q, Q, t)$における正準変換}
        \section{母関数$\Xi(q, Q, t)$における正準変換}
        \section{ラグランジュブラケット}
        \section{正準変換とブラケット}
        \section{微小正準変換}
        \section{正準変換の群構造}
    \chapter{ハミルトン=ヤコビ理論}

    \chapter{解析力学の応用}
        \section{リウヴィルの定理}

\end{document}