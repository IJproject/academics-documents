\documentclass[a4paper]{jsreport}
\usepackage[utf8]{inputenc}
\usepackage{amsmath}

\title{解析力学 - Analytical Mechanics}

\begin{document}

    \maketitle

    \tableofcontents

    \chapter{オーバービュー}
        \section{解析力学とは}
            ここは後々書くヲ(ニュートン力学との違いを中心に)

    \chapter{最小作用の原理とラグランジュ形式}
        \section{一般化座標と一般化運動量}
            解析力学では、\textbf{一般化座標$q_i(t)$}と\textbf{一般化運動量$p_i(t)$}の2つをもとに理論が形成されている。
            この一般化というのは、任意性を持つというニュアンスの接頭辞である。
            つまり、デカルト座標(xyz座標)のようなものに限定されず、任意の座標系について統一的に考えることができるということである。\par
            この後に\textbf{オイラー=ラグランジュ方程式}という解析力学における基礎方程式を導入するが、その方程式では一般化座標と一般化運動量を用いて運動方程式を記述することになる。
            これにより、\textbf{どのような座標系を選択しても、同一の形の方程式で計算をすることができる}という強みを持つ。

        \section{最小作用の原理とラグランジアン}
            $N$個の独立な力学変数$q_i(t)(i=1,2, ..., N)$で記述される$N$自由度系を考える。
            この時、以下のように\textbf{ラグランジアン}という量を定義する。
            \begin{equation}
                L = L(q(t), \dot{q}(t), t)
            \end{equation}
            \begin{align}
                q(t) &= (q_1(t), q_2(t), ..., q_N(t)) \\
                \dot{q}(t) &= (\dot{q}_1(t), \dot{q}_2(t), ..., \dot{q}_N(t))
            \end{align}
            ちなみに、このラグランジアンは直接的に$t$という引数を持つことから、\textbf{陽の時間依存性}を持つという。
            陽な時間依存性を持たなければ、一般化座標と一般化運動量のみでラグランジアンが表せるということになる。\par
            さて、ここで以下のような\textbf{作用$S$}という量を導入する。
            \begin{equation}
                S[q] = \int_{t_1}^{t_2} L(q(t), \dot{q}(t), t) dt
            \end{equation}
            この時、系の運動は$S[q]$が最小になる(\textbf{停留}する)ように決定される。
            これを\textbf{最小作用の原理}という。
            つまり、どのように運動をするのか(どのような経路を辿って運動するのか)は、この作用$S$によって決まる。

        \section{変分法とオイラー=ラグランジュ方程式}
            では、作用$S$が最小になるような経路を求めるためにはどうすればよいか。\par
            まず、$q(t)$から微少量ずれた経路$q(t) + \delta q(t)$を考える。この時の作用の変化量$\delta S[q]$は、$\delta$の2次以上の項を無視し、テイラー展開などを行うことで、
            \begin{align}
                \delta S[q] &= S[q + \delta q] - S[q] \\
                &= \int_{t_1}^{t_2} \left( L(q + \delta q, \dot{q} + \delta \dot{q}, t) - L(q, \dot{q}, t) \right) dt \\
                &= \int_{t_1}^{t_2} dt \left( \frac{\partial L}{\partial q} - \frac{d}{dt} \frac{\partial L}{\partial \dot{q}} \right) \delta q
            \end{align}
            と計算できる。
            行間が大分省略されているが、ご愛嬌ということで。
            ここで、$\delta S[q]$は極値付近(今回は最小付近)の時、変化が緩やかになることから、$\delta S[q] = 0$となると考えるのが自然である。
            つまり、最小作用の原理を満たすような方程式として、
            \begin{equation}
                \frac{\partial L}{\partial q} - \frac{d}{dt} \frac{\partial L}{\partial \dot{q}} = 0
            \end{equation}
            が得られ、これを\textbf{オイラー=ラグランジュ方程式}という。
            さらに、このように微小変化に対する応答(\textbf{変分$\delta S[q]$})が$0$になるような条件で議論を進める手法を\textbf{変分法}という\par
            アインシュタインの縮約ルールに基づいた表記としては、
            \begin{equation}
                \frac{\partial L}{\partial q_i} - \frac{d}{dt} \frac{\partial L}{\partial \dot{q_i}} = 0
            \end{equation}
            となる。

        \section{ラグランジアンの不定性}
            任意関数$G(q(t), t)$について、
            \begin{equation}
                L'(q(t), \dot{q}(t), t) = L(q(t), \dot{q}(t), t) + \frac{d}{dt} G(q(t), t)
            \end{equation}
            というラグランジアン$L'$を考える。
            この$L'$はオイラー=ラグランジュ方程式において$L$と全く同じ解を与える。
            つまり、$L$と$L'$は等価であるといえる。
            これは変分法により、2つのラグランジアンの変分を比較することで示すことができる。
            この性質がラグランジアンの適用範囲を広げることになる。

        \section{ラグランジュの未定乗数法}
            気が向いたら書く

    \chapter{対称性と保存則}
        \section{時間並進の対称性}
            時間並進とは、時間を定数分だけずらすような操作のことである。
            そこで、
            \begin{equation}
                t \to t + t_0
            \end{equation}
            のように変更した場合に、ラグランジアンがどのように応答するかを考える。
            この時、ラグランジアンが陽の時間依存性を持たない場合、$q(t+t_0)$も元のオイラー=ラグランジュ方程式の解になる。
            つまり、どちらの運動も同じラグランジアンで表すことができるということである。\par
            逆に、ラグランジアンが陽の依存性を持つ(=$t$に直接的に依存する)場合、受ける力などが変わってくるため、運動の仕方に違いが生じることは明らかである。
            そのため、同じラグランジアンで表すことができないことは直感的にも理解できる。

        \section{空間並進の対称性}
            空間並進とは、空間座標を一様にずらすような操作のことである。
            「一様に」というのは、座標全体を一定分だけという意味である。
            そこで、
            \begin{align}
                q(t) &\to q(t) + q_0 \\
                \dot{q}(t) &\to \frac{d}{dt} ({q(t)} + q_0) = \dot{q}(t)
            \end{align}
            のように変更した場合に、ラグランジアンがどのように応答するかを考える。
            この時、ざっくりとして表現にはなるが、空間に一様性があれば$q(t) + q_0$も元のオイラー=ラグランジュ方程式の解になる。
            ここでいう空間の一様性というのは、\textbf{座標に依存するような場に関するパラメータが含まれていない}という意味である。
            もちろん場に依存するするようなパラメータが存在するのであれば、働く力などが変わってくることは容易に想像することができる。

        \section{空間回転の対称性}
            空間回転とは、空間座標を原点を中心として回転させるような操作のことである。
            空間回転については、行列式の値が$1$になる回転行列$R$を左から作用させて、
            \begin{equation}
                q(t) \to Rq(t)
            \end{equation}
            となり、これによりラグランジアンがどのように応答するのかを考える。
            この際、\textbf{方向に依存した変化がなければ、空間回転に対して対称性を持つ}ことが分かる。
            例えば、$3$次元空間で$x$軸方向のみに力が働いているような系の場合、空間回転させてしまうと系に対する力の働く向きが変わってしまうため、空間回転に対しては対称ではない。
            返って、中心力による束縛のような系では、空間回転しても分からないため、これは空間回転に対して対称であるといえる。

        \section{ネーターの定理}
            先ほどまでの対称性についての議論を踏まえて、保存則について考える。
            一つの重要な定理として、\textbf{ラグランジアンが対称性を持っている(=不変である)場合、それに対応した保存量が存在する}。
            これを\textbf{ネーターの定理}という。
            つまり、時間並進の対称性を持つ場合、それに対応して何かしらの保存量が存在する。
            これが他の対称性についても成り立つということである。
            対称性の議論から保存則が導出できるということである。\par
            ではこれを数式で追ってみたい。
            保存量$Q$は陽な時間依存性を持たないことから以下のように表される。
            \begin{equation}
                \frac{d}{dt}Q(q, \dot{q}) = 0
            \end{equation}
            ここで、対称性の議論の部分でも行ったような微小変換を、より一般化した場合で考えたい。
            微小変換を表現する量を$F_i^A(q, \dot{q})$として、
            \begin{align}
                q_i(t) &\to q_i(t) + F_i^A(q, \dot{q}) \epsilon_A \\
                \dot{q_i}(t) &\to \dot{q_i}(t) + \frac{d}{dt} F_i^A(q, \dot{q}) \epsilon_A
            \end{align}
            これに対してラグランジアンがどのように応答するのかを考えるために、$\delta L$を計算すると、
            \begin{align}
                \delta L &= \frac{d}{dt} \frac{\partial L(q, \dot{q}, t)}{\partial \dot{q}} F_i^A \\
                &= \frac{d}{dt} Y^A(q, \dot{q}, t) \epsilon_A \\
                &= 0
            \end{align}
            ここから保存量$Q^A$は、
            \begin{equation} \label{eq:3a}
                Q^A = \frac{\partial L(q, \dot{q}, t)}{\partial \dot{q}} F_i^A(q, \dot{q}) - Y^A(q, \dot{q}, t)
            \end{equation}
            で与えられる。
            これにより、\textbf{どのように微小変換させるのか}を表す$F_i^A$と\textbf{ラグランジアンの変分}を表す$\delta L$を求めれば、保存量を導出することができることがわかる。

        \section{エネルギー保存則}
            3.1節で考えた、時間並進の対称性について、ネーターの定理を適用する。
            \begin{align}
                q_i(t) &\to q_i(t + \epsilon_0) = q_i(t) + \dot{q_i}(t)\epsilon_0 \\
                \dot{q_i}(t) &\to \dot{q_i}(t + \epsilon_0) = \dot{q_i}(t) + \ddot{q_i}(t)\epsilon_0
            \end{align}
            まず、ラグランジアンの変分は、全微分を計算することで、
            \begin{equation}
                \delta L = \frac{d}{dt} L(q, \dot{q})\epsilon_0
            \end{equation}
            ネーターの定理(\ref{eq:3a})式との対応関係を考えると、
            \begin{equation} \label{eq:3b}
                E = \frac{\partial L(q, \dot{q})}{\partial \dot{q_i}} \dot{q_i} - L(q, \dot{q})
            \end{equation}
            という保存量$E$を得ることができ、これは一般に\textbf{エネルギー}と呼ばれるものである。
            つまり、\textbf{時間並進について対称(ラグランジアンが不変)の場合、エネルギー保存則が成り立つ}ということである。
            後に出てくるが、これはハミルトニアンの定義式と全く同じ式をしている。

        \section{運動量保存則}
            3.2節で考えた空間並進の対称性に、ネーターの定理を適用する。
            \begin{align}
                q(t) &\to q(t) + \epsilon \\
                \dot{q}(t) &\to \frac{d}{dt} ({q(t)} + q_0) = \dot{q}(t)
            \end{align}
            これとネーターの定理の各量と比較することで、
            \begin{equation}
                P = \sum_{i=1}^{N} \frac{\partial L(q, \dot{q})}{\partial \dot{q_i}}
            \end{equation}
            という保存量$P$を得ることができ、これが\textbf{運動量}と呼ばれるものである。
            保存量であるため、\textbf{空間並進に対して運動量は保存する}ということである。

        \section{角運動量保存則}
            3.3節で考えた空間回転の対称性に、ネーターの定理を適用する。
            \begin{equation}
                q(t) \to q(t) + \delta \phi \times q(t)
            \end{equation}
            これとネーターの定理の各量と比較することで、
            \begin{equation}
                L = \sum_{i=1}^{N} \frac{\partial L(q, \dot{q})}{\partial \dot{q_i}} \delta \phi
            \end{equation}
            という保存量$L$を得ることができ、これが\textbf{角運動量}と呼ばれるものである。
            保存量であるため、\textbf{空間回転に対して角運動量は保存する}ということである。

    \chapter{ハミルトン形式}
        \section{ハミルトニアン}
            一般化座標に対する一般化運動量は以下のように定義される。
            \begin{equation} \label{eq:4a}
                p_i = \frac{\partial L(q, \dot{q}, t)}{\partial \dot{q_i}}
            \end{equation}
            これまでのラグランジュ形式が$(q, \dot{q}, t)$であったことに対して、ハミルトン形式では$(q, p ,t)$に依存するような形式で考える。
            そしてハミルトニアン$H$を以下のように定義する。
            \begin{equation} \label{eq:4b}
                H(q, p ,t) = \sum_{i=1}^N p_i \dot{q_i} - L(q, \dot{q}, t)
            \end{equation}
            これは、(\ref{eq:3b})式と全く同じ形をしていることから、ハミルトニアンはエネルギーに関する情報を持っているといえる。
            ちなみにだが、後述するルジャンドル変換を使用すると、ラグランジアンからハミルトニアンの表式を導出することができる。
            つまり、ラグランジアンと等価であるといえる。

        \section{ハミルトンの運動方程式}
            定義したハミルトニアンの式から、偏微分を計算することにより
            \begin{align}
                \dot{q_i} &= \frac{\partial H(q, p ,t)}{\partial p_i} \label{eq:4c} \\
                \dot{p_i} &= - \frac{\partial H(q, p ,t)}{\partial q_i} \label{eq:4d}
            \end{align}
            という性質を見出すことができる。
            これをハミルトンの運動方程式といい、一階常微分方程式を与え、二階常微分方程式を与えるオイラー=ラグランジュ方程式と等価な方程式である。
            2つが等価であることから、オイラー=ラグランジュ方程式を導出した時と同じように、ハミルトニアンの表式から最小作用の原理から導出することが可能であることは直感的にも理解できる。

        \section{ハミルトニアンの時間微分}
            ハミルトンの運動方程式(\ref{eq:4c})(\ref{eq:4d})を用いることで、ハミルトニアンの時間$t$に関する全微分は、
            \begin{equation} \label{eq:4e}
                \frac{d}{dt} H(q, p ,t) = \frac{\partial H(q, p, t)}{\partial t}
            \end{equation}
            と表される。
            さらに、ハミルトニアンの定義式(\ref{eq:4b})を時間に関して同様の計算をすると、
            \begin{equation} \label{eq:4f}
                \frac{d}{dt} H(q, p ,t) = - \frac{\partial L(q, \dot{q}, t)}{\partial t}
            \end{equation}
            と表される。
            ここまでの議論により、ハミルトニアンの時間に対する応答とラグランジアンとの関係は、
            \begin{equation}
                \frac{d}{dt} H(q, p ,t) = \frac{\partial H(q, p, t)}{\partial t} = - \frac{\partial L(q, \dot{q}, t)}{\partial t}
            \end{equation}
            と表される。
            もちろんだが、これは陽な時間依存性を持つ場合での話で、そうでなければ(\ref{eq:4e})(\ref{eq:4f})の値はどちらも$0$になる。
       
        \section{ポアソンブラケット}
            系の時間発展や変換を表現する量として、
            \begin{equation}
                \{f(q, p, t), g(q, p, t)\} = \sum_{i=1}^N \left(\frac{\partial f}{\partial q_i}\frac{\partial g}{\partial p_i} - \frac{\partial f}{\partial p_i}\frac{\partial g}{\partial q_i}\right)
            \end{equation}
            で表される\textbf{ポアソンブラケット}を定義する。
            ポアソンブラケットの定義式を見れば、線形性が成り立つことは自明なので、それに伴った性質が現れる。\par
            さらに、証明は省くが、
            \begin{equation}
                \{ f, \{ g, h \} \} + \{ g \{ h, f \} \} + \{ h \{ f, g \} \} = 0
            \end{equation}
            というような\textbf{ヤコビ恒等式}が成り立つ。

        \section{ポアソンブラケットと物理量}
            ある物理量$f(q, p ,t)$に対して、$\{ {f, H} \}$は系のハミルトニアン$H$の元での$f$の時間発展を表す量であることを念頭に置いて、時間$t$での全微分を考えると、
            \begin{equation}
                \frac{df}{dt} = \{f, H\} + \frac{\partial f}{\partial t}
            \end{equation}
            と表現することができる。
            ここから、物理量$f$が陽な時間依存性を持たない場合、
            \begin{equation}
                \frac{df}{dt} = \{f, H\}
            \end{equation}
            と全微分をポアソンブラケットの表式だけで表すことができる。\par
            さらに、物理量$f$と一般化座標、一般化運動量の間では、
            \begin{align}
                \{ q_i, f \} &= \frac{\partial f}{\partial p_i} \\
                \{ p_i, f \} &= - \frac{\partial f}{\partial q_i}
            \end{align}
            という関係が成り立つことから、ポアソンブラケットは、微小変化に対して物理量がどのように変化するかを表す量であるといえる。
            これから基本ポアソンブラケットと呼ばれるものを導出することができる。


    \chapter{正準変換}
        \section{ルジャンドル変換}
            ルジャンドル変換というのは、変数の変換をするような操作である。
            ただ無制限にということではなく、変換先の変数は偏微分を使用してできるものである。
            ある関数$f(x, y)$を考える。
            ここで、
            \begin{equation}
                u = \frac{\partial f}{\partial x} \quad v = \frac{\partial f}{\partial y}
            \end{equation}
            という$u$、$v$を用意すれば、系の構造は不変のままで変数を変換することができる。
            \begin{align}
                g(u, y) &= f(x, y) - ux \\
                h(u, v) &= f(x, y) - ux - vy
            \end{align}
            例えばこれをラグランジアン$(q, p, t)$に対して行うことで、ハミルトニアン$(q, p, t)$に変換することができる。

        \section{正準変換}
            これまで扱ってきた変数$(q, p)$は\textbf{正準変数}と呼ばれている。
            これを元に、以下のような形式で新しい正準変数$(Q, P)$を作成することを\textbf{正準変換}という。
            この時、この正準変数におけるハミルトニアンは$K(Q, P, t)$と表される。
            \begin{equation}
                Q_i = Q_i(q, p, t) \quad P_i = P_i(q, p, t)
            \end{equation}
            もちろん無条件に変換していいわけではなく、\textbf{ハミルトンの運動方程式を満たすこと}や\textbf{ポアソンブラケットを不変に保つこと}などの条件がある。
            つまり、表現している物理系の性質を変えない範囲内で、正準変数を変換する操作であるといえる。
            これにより、運動方程式を解きやすい形に変換した上で解くことができるなどの恩恵を受けることができる。

        \section{母関数$F(q, Q, t)$における正準変換}
            新しい変数$(Q, P, t)$への変換に関して、
            \begin{equation}
                \sum_{i} p_i \dot{q_i} - H(q, p, t) = \sum_{i} P_i \dot{Q_i} - K(Q, P, t) + \frac{d}{dt} F(q, Q, t)
            \end{equation}
            を満たすような$F(q, Q, t)$が存在すれば、これは正準変換であり$K(Q, P, t)$を新ハミルトニアン、$F(q, Q, t)$を母関数と呼ぶ。
            この母関数$F(q, Q, t)$の時間に関する全微分を考え、項の比較をすることで、
            \begin{equation} \label{eq:5a}
                p_i = \frac{\partial F(q, Q, t)}{\partial q_i} \quad P_i = - \frac{\partial F(q, Q, t)}{\partial Q_i} \quad K = H + \frac{\partial F(q, Q, t)}{\partial t}
            \end{equation}
            という関係式を得ることができる。
            これにより、母関数をひとつ与えれば正準変換が定義されるということがわかる。
            この正準変換を基本形として、ルジャンドル変換を使用することでさらに正準変数の組み替えをすることが可能になる。

        \section{母関数$\Phi(q, P, t)$における正準変換}
            5.3節で得られた母関数をルジャンドル変換($Q \to P$)して、正準変数$(q, P, t)$を作る。
            変換式は、
            \begin{equation}
                \Phi(q, P, t) = F(q, Q, t) + \sum_{i} P_i Q_i
            \end{equation}
            と表される。
            $F(q, Q, t)$を計算した時と同じ要領で、全微分の計算をすることで、以下のような関係式を得ることができる。
            \begin{equation} \label{eq:5b}
                p_i = \frac{\partial \Phi(q, P, t)}{\partial q_i} \quad Q_i = \frac{\partial \Phi(q, P, t)}{\partial P_i} \quad K = H + \frac{\partial \Phi(q, P, t)}{\partial t}
            \end{equation}

        \section{母関数$\Psi(p, Q, t)$における正準変換}
            5.4節と計算の流れは全く同じ。
            \begin{equation}
                \Psi(p, Q, t) = F(q, Q, t) - \sum_{i} p_i q_i
            \end{equation}
            とし、全微分を考えることにより、
            \begin{equation} \label{eq:5c}
                q_i = - \frac{\partial \Psi(p, Q, t)}{\partial p_i} \quad P_i = - \frac{\partial \Psi(p, Q, t)}{\partial Q_i} \quad K = H + \frac{\partial \Psi(p, Q, t)}{\partial t}
            \end{equation}

        \section{母関数$\Xi(p, P, t)$における正準変換}
            こちらも全く同じで、同時にルジャンドル変換を行えばよい。
            \begin{equation}
                \Xi(p, P, t) = F(q, Q, t) + \sum_{i} P_i Q_i - \sum_{i} p_i q_i
            \end{equation}
            とし、全微分を考えることにより、
            \begin{equation} \label{eq:5d}
                q_i = - \frac{\partial \Xi(p, P, t)}{\partial p_i} \quad Q_i = \frac{\partial \Xi(p, P, t)}{\partial P_i} \quad K = H + \frac{\partial \Xi(p, P, t)}{\partial t}
            \end{equation}
        
        \section{正準変換とポアソンブラケット}
            正準変換の一つの特徴として、\textbf{ポアソンブラケットを不変に保つ}という性質があった。
            つまり、
            \begin{equation}
                \{ f, g \}_{q, p} = \{ f, g \}_{Q, P}
            \end{equation}
            という等式が成り立つ。
            この結論の主張は、\textbf{正準変換に対して、力学的構造が変化しない}ことや、\textbf{時間依存性が変わらない(=保存量を引き継ぐ)}ことなどが挙げられる。

        \section{微小正準変換}
            正準変数の微小変化を新たな正準変数とすることを\textbf{微小正準変換}という。
            \begin{equation}
                Q_i = q_i + \delta q_i \quad P_i = p_i + \delta p_i
            \end{equation} 
            この微小変化を、$q_i \to Q_i \quad p_i \to P_i$という\textbf{恒等変換}を用いて考える。
            この変換を定義できる母関数$\Phi (q, P, t)$を用いた恒等変換に微小変化分$\epsilon G$を加えて、
            \begin{equation}
                \Phi (q, P, t) = \sum_{i} q_iP_i + \epsilon G(q, P, t)
            \end{equation}
            という母関数を使用する。これと(\ref{eq:5c})式を使うことで、
            \begin{equation}
                \delta q_i = \epsilon \frac{\partial G(q, p, t)}{\partial p_i} \quad \delta p_i = - \epsilon \frac{\partial G(q, p, t)}{\partial q_i}
            \end{equation}
            これより、この$G$を微小変換の母関数という。
            この$G$にエネルギーや運動量のような保存量を選べば、ラグランジュ形式の議論で出てきた\textbf{対称性}の式が現れる。
            これはラグランジュ形式とハミルトン形式が等価であるからこそ得られる結果なように思える。

        \section{正準変換の群構造}
            正準変換は系の力学的な構造を不変に保ったままの状態で正準変数を変換することができる。
            よって、変数をどれだけ変換しても系の構造は変化しない。
            つまり、変換を連続的に行う(\textbf{合成}する)ことが可能なのは自明である。
            さらに、正準変換を逆に辿る操作ができることも自明ではないが直感的に理解できる。
            このことから、単位元にあたる恒等変換や逆元にあたる正準変換の逆があることなどから、正準変換は群構造を持つことがわかる。

    \chapter{ハミルトン=ヤコビ理論}

    \chapter{解析力学の応用}
        \section{位相空間軌跡}
        \section{リウヴィルの定理}

\end{document}