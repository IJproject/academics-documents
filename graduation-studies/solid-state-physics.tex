\documentclass[a4paper]{jsreport}
\usepackage[utf8]{inputenc}
\usepackage{amsmath}
\usepackage{physics}

\title{固体物理学 - Solid State Physics}

\begin{document}

    \maketitle

    \tableofcontents

    \chapter{オーバービュー}
    \chapter{結晶構造}
        \section{格子ベクトル}
            ブラべ格子、格子ベクトルの表式\par
            結晶は膨大な数の原子が並んで形成されている。
            その原子は規則正しく並んでいる(とする)わけであるが、どの原子から見ても景色が同じような点の集まりを\textbf{ブラべ格子}という。
            表現が分かりづらいので少し噛み砕くと、どのような周期性を持って原子が並んでいるのかを判別するためにピックアップした原子群といったようなイメージである。
            基本的な固体物理学のアプローチとして、周期構造の中の一つ(\textbf{単位格子})で解析をし、それを周期的境界条件で実際のスケールに拡張する。
            そのプロセスの中で、ブラべ格子を考えることは非常に意義のあることである。\par
            その周期的境界条件をどのように表現するのか。
            ここで、\textbf{格子ベクトル}を\textbf{基本並進ベクトル}と呼ばれる$\boldsymbol{a_i}$を使用して、
            \begin{equation} \label{eq:21a}
                \boldsymbol{T} = n_1 \boldsymbol{a_1} + n_2 \boldsymbol{a_2} + n_3 \boldsymbol{a_3}
            \end{equation}
            と定義し、格子点の位置$\boldsymbol{r}$、$\boldsymbol{r'}$を用いて、
            \begin{equation}
                \boldsymbol{r'} = \boldsymbol{r} + \boldsymbol{T}
            \end{equation}
            となることとする。
            この格子ベクトルを使えば、周期性を表現することができる。

        \section{結晶面}
            結晶(格子)の面を表現するには\textbf{ミラー指数}と呼ばれる量を使用し、$(hkl)$というように表記する。
            ここでは、基本並進ベクトル$\boldsymbol{a_i}$を使って、その軸上での点$0$から$1$の範囲で指定し、その逆数をそれぞれ取ったものがそれらの点を繋いでできる面がミラー指数で表される結晶面となる。
            文章だけで説明することは難しいので、ここではこれぐらいの記述に留めておく。
            ちなみにミラー指数は、逆格子空間での記述に対応させており、表記として$(hkl) = (m_1m_2m_3)$と、逆格子ベクトルの係数を順に並べるように定義されている。

        \section{逆格子ベクトル}
            実空間での解析よりも、\textbf{逆格子空間}(波数空間)と呼ばれる、実空間と逆の次元を持つような空間での解析の方が見通しが良い。
            理由は様々であるが、運動量や分散との対応が分かりやすくなるなどがある。\par
            実空間での周期性を損なうことがないように、逆格子空間を定義していきたい。
            任意の物理量$f(\boldsymbol{r})$について、実格子の周期性から
            \begin{equation} \label{eq:23a}
                f(\boldsymbol{r} + \boldsymbol{T}) = f(\boldsymbol{r})
            \end{equation}
            が成り立つ。
            ここで周期性に相性が良いフーリエ級数展開を利用することにより、
            \begin{equation} \label{eq:23b}
                f(\boldsymbol{r}) = \sum_{m} A_m \exp(i\boldsymbol{G_m} \cdot \boldsymbol{r})
            \end{equation}
            と表現することができ、ここで現れた$\boldsymbol{G_m}$が逆格子ベクトルに当たるものである。
            平面波の表式と見比べると、波数$\boldsymbol{k}$に当たる部分が$\boldsymbol{G_m}$になっていることから、逆格子空間は波数空間と呼ばれることがあるのだろう。
            以上までの議論を踏まえて、\textbf{逆格子基本並進ベクトル}を
            \begin{equation}
                \boldsymbol{b_i} = 2 \pi \frac{\boldsymbol{a_j} \times \boldsymbol{a_k}}{\boldsymbol{a_i} \cdot \boldsymbol{a_j} \times \boldsymbol{a_k}}
            \end{equation}
            と定義し、これを用いて\textbf{逆格子ベクトル}を、
            \begin{equation}
                \boldsymbol{G_m} = m_1 \boldsymbol{b_1} + m_2 \boldsymbol{b_2} + m_3 \boldsymbol{b_3}
            \end{equation}
            と表すことにする。
            これを実際に(\ref{eq:23b})式に代入してみると、(\ref{eq:23a})式を満たしていることが分かる。
            ここでは、
            \begin{equation} \label{eq:23f}
                \boldsymbol{a_i} \cdot \boldsymbol{b_j} = 2 \pi \delta_{i,j}
            \end{equation}
            という関係式が成り立っており、実空間の基本並進ベクトルと長さの情報に関するものだけが逆になったような、密接すぎる関係があることは見て取れる。
            実際、実格子空間と逆格子空間は、フーリエ変換と逆フーリエ変換により行き来することができる。
            フーリエ変換は、重ね合わさった波を波数ごとに分離する(波束 → 単一波)に対応し、逆フーリエ変換はその逆の操作に対応する。
            このことを考えると、(\ref{eq:23f})式で$2\pi$の因子が出てくることも直感的に理解することができる。

        \section{回折条件}
            位置$\boldsymbol{r}$にある体積素片での回折について考えたい。
            入射波と散乱波の波数をそれぞれ$\boldsymbol{k}$、$\boldsymbol{k'}$と表現することにすると、位相差による因子は、$\exp{i(\boldsymbol{k} - \boldsymbol{k'}) \cdot \boldsymbol{r}}$と表される。
            この位相差の因子を見て、強めあう条件は、波数差が周期性と一致するときであると言えるので、
            \begin{equation}
                \boldsymbol{k} - \boldsymbol{k'} = - \Delta k = \boldsymbol{G}
            \end{equation}
            となるときである。
            ちなみに、この$- \Delta k$は\textbf{散乱ベクトル}と呼ばれる量である。
            この散乱が\textbf{弾性散乱}である場合、式変形をしていくことで、
            \begin{equation} \label{eq:25b}
                2 \boldsymbol{k} \cdot \boldsymbol{G} = G^2
            \end{equation}
            という表式が現れる。
            ここまでの議論は、強め合う条件に関するもので、最後の表式は\textbf{ブラッグの条件}の別の表し方であると言える。

        \section{ブリルアンゾーン}
            実空間の単位格子に\textbf{ウィグナー・サイツ・セル}と呼ばれるものがある。
            これは各格子点との中間面に囲まれた領域のことで、格子内に原子をただ1つだけ含む最小の格子(単位格子)である。\par
            逆格子空間におけるウィグナー・サイツ・セルに当たるものが\textbf{ブリルアンゾーン}である。
            (\ref{eq:25b})式を利用して、両辺を$4$で割ることで、
            \begin{equation}
                \boldsymbol{k} \cdot \left(\frac{1}{2} \boldsymbol{G}\right) = \left(\frac{1}{2} G\right)^2
            \end{equation}
            という表式が成り立ち、これがブリルアンゾーンを表す。
            このブリルアンゾーンは、中間面によって囲まれる空間の取り方に任意性があり、体積が小さい順に\textbf{第一ブリルアンゾーン}、\textbf{第二ブリルアンゾーン}、、と呼ばれる。
            もちろんのこと、境界は(\ref{eq:25b})式を満たす、つまりブラッグの条件を満たすので回折現象により強め合う。

        \section{構造因子}
            散乱された波の振幅\textbf{(散乱振幅$F$)}は、その位置の電子密度$n(\boldsymbol{r})$に依存し、
            \begin{equation}
                F = N \int dV n(\boldsymbol{r}) \exp(-i\boldsymbol{G} \cdot \boldsymbol{r}) = NS_G
            \end{equation}
            と表される。
            ちなみに、振幅の2乗が強度に当たる。
            これより、電子間の相互作用などを踏まえて、導出過程は省略するが、
            \begin{equation}
                S_G = \sum_{j}f_j \exp(-i\boldsymbol{G} \cdot \boldsymbol{r_j})
            \end{equation}
            となり、この$S_G$を\textbf{構造因子}、$f_j$を\textbf{原子構造因子}と呼ぶ。
            $f_j$については、原子(の状態)ごとに決まった値を持つものである。
            この$S_G$は、結晶構造を如実に表すものであり、結晶にどこに光を当てるとどれほど散乱するのかなどをすることができ、例えば散乱強度が$0$になるようなパラメータでは弱め合っているということが分かる。




    \chapter{結合}
        \section{結合とは}
            原子はなぜ結合するのか。
            その理由はシンプルで、結合した方が安定するからである。
            それを定量的に見ていきたいのだが、原子が完全に自由な時と、結合して結晶になっている時のエネルギー差を\textbf{結合エネルギー}と呼ぶ。
            この結合エネルギーが大きいほど、結合しやすいと言える。\par
            2原子間の相互作用を表すポテンシャルエネルギー$V(R)$を使って、
            \begin{equation}
                F(R) = - \frac{dV(R)}{dR}
            \end{equation}
            は\textbf{原子間力}を与える。
            このポテンシャルエネルギー$V(R)$は、ある$R_0$で極小値を取り、その$R_0$が安定の位置となり、そのときの原子間力は$0$になる。\par

        \section{分子結合}
            2つの水素原子の結合について考える。
            それぞれの原子の波動関数を$\varphi_1$, $\varphi_2$と表現することにすると、分子の波動関数は、
            \begin{equation}
                    \phi = C_1 \varphi_1 + C_2 \varphi_2
            \end{equation}
            と一次結合で表される。
            この手法のことを\textbf{LCAO法}という。
            この波動関数のエネルギー準位を考えると、元のエネルギー準位よりも低い準位(安定的で結合に寄与する価電子帯)と高い準位(非安定的で電気伝導に寄与する伝導帯)の2つに分裂する。
            前者を\textbf{結合性軌道}、後者を\textbf{反結合性軌道}という。
            実際、それぞれの波動関数を導出し、それを2乗することにより電子の存在確率を計算すると、結合性たる所以と反結合性たる所以を体感することができるが、ここでは時間の関係でカットする。
            ちなみに、この結合性軌道にある電子は\textbf{価電子帯}の電子に対応し、反対に反結合性軌道にある原子は\textbf{伝導帯}にある電子に対応する。

        \section{イオン結合}
            陽イオンになりやすい原子と陰イオンになりやすい原子はしばしば\textbf{イオン結合}と呼ばれる、後述の共有結合とは異なる結合をすることがある。
            この結合が安定するのは、陽イオンにするために必要なエネルギー(イオン化エネルギー)よりも陰イオンになった時に放出されるエネルギー(電子親和力)とクーロンポテンシャルの和の方が大きいからである。
            つまり、
            \begin{equation}
                (イオン化エネルギー) - (電子親和力) - (クーロンポテンシャル) < 0
            \end{equation}
            となり、これは孤立原子の状態よりも安定であることを表している。
            イオン結合における具体的な表式について述べてい。
            任意のイオン$i$, $j$には、\textbf{ボルン・マイヤーポテンシャル}という斥力ポテンシャルがあり、
            \begin{equation}
                U_{ij} = \lambda exp\left( - \frac{r_{ij}}{\rho} \right)
            \end{equation}
            と表される。
            最近接イオン間距離を$R$とし、斥力はそれとしか相互作用しないとし、
            \begin{equation}
                r_{ij} = p_{ij}R
            \end{equation}
            と書けることを踏まえると、格子としてのポテンシャルは、
            \begin{equation}
                U_tot = NU_i = NZ \lambda exp\left( - \frac{r_{ij}}{\rho} \right) - \frac{\alpha q^2 N}{4 \pi \epsilon_0 R }
            \end{equation}
            と表される。
            この式中の$\alpha$は\textbf{マーデルング定数}と呼ばれる量で、
            \begin{equation}
                \alpha = \sum_{i,j} \frac{(\pm)}{p_{ij}}
            \end{equation}
            と表され、

        \section{共有結合}
        \section{金属結合}
        \section{ファン・デル・ワールス結合}
        \section{水素結合}

    \chapter{フォノン}
        \section{位相速度と群速度}
            波の伝わる速度には、単一の波の場合と波束の場合の2種類ある。
            単一の波の場合、その波の伝わる速度を\textbf{位相速度}といい、
            \begin{equation}
                v_p = \frac{\omega}{k}
            \end{equation}
            と表される。
            それに対して、波束(重ね合わされた波)の場合、その伝わる(ように見える)速度を\textbf{群速度}といい、
            \begin{equation}
                v_g = \frac{d\omega}{dk}
            \end{equation}
            と表される。
            当たり前の議論ではあるが、位相速度と群速度は一般には一致しない。
            群速度を構成する位相速度には振れ幅(分散)があるからである。

        \section{一次元単原子格子の格子振動}
            格子が単一の原子で構成されている場合を考える。
            格子の振動のモデルはしばしば原子間にバネを設けたモデルで表現される。
            今回もそのアプローチをとることとして、全ての原子が同じ角振動数で振動する(基準モード)として考える。
            詳しい導出の過程は省略するが、角振動数$\omega$は、
            \begin{equation}
                \omega(k) = \sqrt{\frac{4K}{M}} \left|\sin{\frac{ka}{2}} \right|
            \end{equation}
            と表され、この関係を\textbf{分散関係}という。
            この言われの所以は、この分散関係によって位相速度や群速度を導出することが可能であることから来ており、それらを決定することで波動の伝搬特性を知ることができるためである。
            長波長の極限では、
            \begin{equation}
                \omega = a \sqrt{\frac{K}{M}}k
            \end{equation}
            と表され、角周波数は波数に比例する式が得られる。
            この性質を、連続媒質中を伝わる音波(弾性波)と同じ性質を満たすことから\textbf{音響モード}という。
            イメージとしては、格子間弾性的相互作用的なものである。

        \section{一次元二原子格子の格子振動}
            格子内の原子を一個増やして考えてみたい。
            実際問題、格子内の原子の数は複数個にして考えることの方が、対称性などの観点から有益なことが多い。
            詳しい導出過程は省略するが、分散関係は
            \begin{equation}
                \omega_{\pm}^2(k) = K \left(\frac{1}{M_1} + \frac{1}{M_2} \right) \pm K \sqrt{\left(\frac{1}{M_1} + \frac{1}{M_2} \right)^2 - \frac{4\sin^2{ka}}{M_1M_2}}
            \end{equation}
            と表される。
            これのうち、$\omega_-$は音響モードを表しており、長波長領域では
            \begin{equation}
                \omega(k) = \sqrt{\frac{2K}{M_1 + M_2}}ka
            \end{equation}
            と表され、単原子格子の場合と同様に波数$k$に依存するような関数系を持つ。
            $\omega_+$は、電磁波の吸収などのような光学的な性質から\textbf{光学モード}と呼ばれる。
            実際にそれぞれの分散関係を図示すると分かるのだが、ここでも省略してしまうことにする。

        \section{三次元格子の格子振動}
        \section{フォノンによる量子化}
            N個の粒子から構成される3次元結晶の場合、自由度は3Nで、これは3N個の独立な調和振動子があるとみなすことができる。
            これにより、
            \begin{equation}
                E = \sum_{\boldsymbol{k}, s} \left(n_{\boldsymbol{k},s} + \frac{1}{2}\right) \hbar \omega_s{\boldsymbol{k}}
            \end{equation}
            とエネルギーを表現することができる。
            この式の理解としては、まず3N個ある各モードに対して量子化された粒子(\textbf{フォノン})が存在する。
            このフォノンというのは、状態の励起具合を表していて、このフォノンの数が多い(振動の腹が多い)ほどエネルギーが高いということに対応する。
            各モードの合計のエネルギーが、系のエネルギーになるということである。

        \section{格子比熱}

        \section{熱膨張}
        \section{熱伝導}

    \chapter{自由電子フェルミ気体}
        \section{フェルミ・ディラック分布}
            電子は、\textbf{フェルミオン}と呼ばれる粒子である。
            フェルミオンというのは、同じエネルギー量子状態に2つ以上の粒子が入ることができないという性質を持つ粒子のことである。
            \textbf{パウリの排他原理}が成り立つのは、電子がフェルミオンだからである。\par
             N電子系の平衡状態において、エネルギー$\epsilon$の状態にある確率を求める。
             天下り的になってしまうが、量子統計力学的なアプローチから、
             \begin{equation}
                f(\epsilon) = \frac{1}{\exp[\beta(\epsilon - \mu)] + 1}
             \end{equation}
             と表され、これを\textbf{フェルミ・ディラック分布}という。
             これと後述の状態密度を用いることで、特定の量子状態を持つ電子の粒子数を計算することができる。

        \section{3次元の自由電子気体}
             ここでは、3次元の箱に閉じ込められた自由電子について考える。
             計算の過程は省略してしまうが、波動関数が
             \begin{equation}
                \psi_{\boldsymbol{k}} = \frac{1}{\sqrt{V}} \exp(i\boldsymbol{k} \cdot \boldsymbol{r})
             \end{equation}
             と表される(周期的境界条件を満たす)ことを用いて、エネルギーと運動量は
             \begin{equation}
                \epsilon_{\boldsymbol{k}} = \frac{\hbar^2}{2m} \left(k_x^2 + k_y^2 + k_z^2 \right)
             \end{equation}
             \begin{equation}
                \boldsymbol{p}\psi_{\boldsymbol{k}} = \hbar \boldsymbol{k}\psi_{\boldsymbol{k}}
             \end{equation}
             と表される。
             ここからエネルギーに注目して議論を進めていく。
             電子をエネルギー準位の低い方から順々に詰めていった際、その基底状態における最大エネルギーを\textbf{フェルミエネルギー}という。
             このフェルミエネルギーの波数を$k_F$と表すことにすると、フェルミエネルギー$\epsilon_F$は
             \begin{equation}
                \epsilon_F = \frac{\hbar^2}{2m} k_F^2
             \end{equation}
             と表されることになる。

        \section{状態密度}
             \textbf{逆格子空間において、各格子点は量子状態に対応する}。
             これは単純に、波数ベクトルがエネルギー準位を決定するためであり、3次元の自由電子気体での議論から理解できることである。
             よって、逆格子空間においては単位格子の体積$2\pi/L$に、一つの量子状態(電子のスピンを考慮していないので、一つの量子状態にスピン自由度2)が存在することになるので、電子の総数を
             \begin{equation}
                N = 2 \times \frac{4\pi {k_F}^3 / 3}{(2\pi/L)^3} = \frac{V}{3\pi^2}{k_F}^3
             \end{equation}
             と表すことができる。
             ここで、$k_F$を半径とした球を\textbf{フェルミ球}といい、基底状態において量子状態がこの球内に収まっていることを表す。
             これにより、フェルミエネルギーを粒子数を用いて、
            \begin{equation}
                \epsilon_F = \frac{\hbar^2}{2m} \left(\frac{3\pi^2 N}{V}\right)^{2/3}
            \end{equation}
            と表され、これは粒子数を増やすとフェルミエネルギーが上昇するという直感にあった結果を得ることができる。
            この直感というのは、電子は低い準位からつまっていくわけだが、電子が増えることで、どの電子は空いている準位(フェルミエネルギー近傍)に入るため、フェルミエネルギーは上昇するだろうということである。
            さらに粒子数はエネルギーを用いて、
            \begin{equation}
                N = \frac{V}{3\pi^2} \left(\frac{2m\epsilon}{\hbar^2} \right) ^ {3/2}
            \end{equation}
            と表される。
            これより、エネルギー変化に対する粒子数の変化を考えることができ、
            \begin{equation}
                D(\epsilon) = \frac{dN}{d\epsilon} = \frac{V}{2\pi^2} \left(\frac{2m}{\hbar^2} \right) ^ {3/2} \epsilon^{1/2}
            \end{equation}
            で表される$D(\epsilon)$を\textbf{状態密度}という。
            これは特定のエネルギーを持つ電子がどれだけあるかを表現した量である。
            ここまでの議論を踏まえると、基底状態における状態密度$D(\epsilon)$に、そこに電子がある確率$f(\epsilon)$を掛けて積分すれば、粒子数になることは直感的にも理解できる。
            \begin{equation}
                N = \int D(\epsilon) f(\epsilon) d\epsilon
            \end{equation}
            適切な定義域を設定する必要があるので、ここでは不定積分にしている。

        \section{電子気体の比熱}


    \chapter{エネルギーバンド}
        \section{結晶内の電子}
            結晶内の電子の状態を記述する方法として、\textbf{1電子近似}という方法を採用する。
            これは、電子間の相互作用がないものとして扱う方法である。
            これにより、比較的シンプルなシュレーディンガー方程式を解くことができる。
            結晶内の電子は、
            \begin{equation}
                \left[ -\frac{\hbar^2}{2m} \nabla^2 + V(\boldsymbol{r}) \right] \psi(\boldsymbol{r}) = E \psi(\boldsymbol{r})
            \end{equation}
            というシュレーディンガー方程式に従い、式中の$V(\boldsymbol{r})$を\textbf{結晶ポテンシャル}といい、格子が電子に与える相互作用ポテンシャルである。
            このポテンシャルは周期的であるので、基本並進ベクトル(格子ベクトル)$\boldsymbol{R_n}$を用いて、
            \begin{equation}
                V(\boldsymbol{r}) = V(\boldsymbol{r} + \boldsymbol{R_n})
            \end{equation}
            と表される。
            このポテンシャルで表されるような系のシュレーディンガー方程式の解は、\textbf{ブロッホの定理}により、
            \begin{equation}
                \psi_{\boldsymbol{k}}(\boldsymbol{r}) = \exp(i\boldsymbol{k} \cdot \boldsymbol{r}) u_{\boldsymbol{k}}(\boldsymbol{r})
            \end{equation}
            という形で書け、
            \begin{equation}
                u_{\boldsymbol{k}}(\boldsymbol{r} + \boldsymbol{R_n}) = u_{\boldsymbol{k}}(\boldsymbol{r})
            \end{equation}
            という関係が成り立つ。
            つまり、周期性を満たすような波動関数を得られるという定理である。
            そのような電子を\textbf{ブロッホ電子}と呼んだり、そのような状態を\textbf{ブロッホ状態}という。


        \section{空格子近似}
            エネルギーバンドを考える際の第0近似として、\textbf{空格子近似}と呼ばれるものがある。
            この近似方法は、\textbf{格子の周期性を保っていながら、結晶ポテンシャルを持たない}という理想的なモデルである。
            つまり、これは自由電子としての近似にあたり、適用されるシュレーディンガー方程式は、
            \begin{equation}
                -\frac{\hbar^2}{2m} \nabla^2 \psi = E \psi
            \end{equation}
            であり、周期性が保たれていることから、
            \begin{equation}
                \psi_{\boldsymbol{k}}(\boldsymbol{r}) = \exp(i\boldsymbol{k} \cdot \boldsymbol{r}) \exp(i\boldsymbol{G_m} \cdot \boldsymbol{r})
            \end{equation}
            となり、エネルギーは、
            \begin{equation}
                E(\boldsymbol{k}) = \frac{\hbar^2}{2m} \left|\boldsymbol{k} + \boldsymbol{G_m} \right| ^2
            \end{equation}
            と表される。
            このモデルでは、バンドギャップは生成されないことから、絶縁体や半導体を表現するようなモデルとしては不適であることがエネルギーの表式を見ることからも分かる。
            (そもそも自由電子的な議論をする空格子近似のモデルは金属的であり、金属のバンド構造を表現することに向いた近似)

        \section{結晶ポテンシャルを含めた近似}
            実際の格子では、結晶ポテンシャルが電子に対して働くことで、エネルギー状態に変化が生まれる。
            波動関数を自由電子の解である平面波で展開すると、
            \begin{equation}
                \psi(\boldsymbol{r}) = \frac{1}{\sqrt{L}} \sum_{\boldsymbol{k}} C_{\boldsymbol{k}} e^{i\boldsymbol{k} \cdot \boldsymbol{r}}
            \end{equation}

            この辺、書くのめんどくせぇな

        \section{擬ポテンシャル}

        \section{強束縛下の電子}

        \section{電子の動力学}
            量子力学において、
            % 結晶運動量、加速定理、有効質量

        \section{電子と正孔}

        \section{フェルミ面}

    \chapter{電気的性質}
        \section{古典的電気伝導}
        \section{ホール効果}
        \section{電気抵抗}
        \section{波数空間での電気伝導}
        \section{ボルツマンの輸送方程式}

    \chapter{光学的性質}
        \section{物質の電磁気学}
        \section{金属の光学的性質}
        \section{半導体と絶縁体の光学的性質}

    \chapter{磁性}
        \section{磁気モーメントと磁化}
            物質の磁気的な性質は、物質内電子の持つ\textbf{磁気モーメント}という量に深く関係している。
            議論を進めていく上で、磁荷(モノポール)の存在を仮定する。
            その磁荷を$q_m$とすると、正負の磁荷の間の磁気モーメントは、
            \begin{equation}
                \boldsymbol{\mu}_m = q_m \boldsymbol{d}
            \end{equation}
            と表される。
            ただ、実際に磁荷は存在していないと考えられており、磁気モーメントは電子の周回運動により定義することの方が適している。
            この電子の周回運動による磁気モーメントは、
            \begin{equation}
                \boldsymbol{\mu}_m = - \frac{e}{2m} \boldsymbol{l}
            \end{equation}
            と表される。
            これに対して、スピンによる磁気モーメントは、\textbf{ランデのg因子}と呼ばれる量を導入して、
            \begin{equation}
                \boldsymbol{\mu}_s = g \left(- \frac{e}{2m} \right) \boldsymbol{s}
            \end{equation}
            と表現される。
            ハミルトニアン$H$が
            \begin{equation}
                H = - \mu \cdot B
            \end{equation}
            で表されることを考えると、これらの磁気モーメントが磁場をかけた際にエネルギー準位に変化が出るのは一目瞭然である。
            実際、磁場をかけていない状態ではエネルギー準位は縮退しており、磁場をかけることでそのエネルギー準位は量子状態に基づいて分裂する。
            それを\textbf{ゼーマン効果}といい、詳しい表式は省略するが、そこから\textbf{ラーモア周波数}という量も求めることができ、磁場をかけた際に磁気モーメントを持つ電子が磁場に運動状態を拘束されるような現象(歳差運動)の角周波数を得ることができる。\par
            最後に磁化について述べておくと、物質中の磁束密度は、
            \begin{equation}
                \boldsymbol{B} = \mu_0 (\boldsymbol{H} + \boldsymbol{M})
            \end{equation}
            と表され、この$\boldsymbol{M}$は磁化ベクトルと呼ばれ、磁化率を$\chi$で表すことにすると、
            \begin{equation}
                \boldsymbol{M} = \chi \boldsymbol{H}
            \end{equation}
            と表されるので、物質中の磁束密度は、
            \begin{equation}
                \boldsymbol{B} = \mu_0 (1 + \chi) \boldsymbol{H}
            \end{equation}
            と表されることになる。

        \section{常磁性}
            不完全殻を持つような原子やイオンのことを、\textbf{磁性原子(磁性イオン)}という。
            完全殻の磁気モーメントは平均するとゼロになるが、不完全殻の磁気モーメントは平均してゼロにならない。
            このことから、磁場をかけると全体としてわずかに磁化し、この性質を\textbf{常磁性}という。
            ただし、高温の場合は熱運動が支配的であるため、磁化がなくなることがある。
            常磁性体は、磁場をかけることで$2J+1$個の縮退が解け、$m_J$で指定される状態の確率は統計力学的な議論から、
            \begin{equation}
                P(m_J) = \frac{\exp(- E_{m_J} B)}{\sum\limits_{m_J=-J}^{J} \exp(- E_{m_J} B)}
            \end{equation}
            と表される。
            この値から磁化を求めることができ、N個の原子のある系の場合、
            \begin{equation}
                M = - N \mu_B g_J \sum\limits_{m_J=-J}^{J} m_J P(m_J)
            \end{equation}
            と表される。
            この級数をブリュアン関数を考えることで計算することができ、詳しい計算の過程は省略するが、
            \begin{equation}
                M = \frac{N \mu_B^2 g_J^2 J(J+1) \beta}{3} B
            \end{equation}
            と表され、磁化率は
            \begin{equation}
                \chi = \frac{N \mu_0 \mu_B^2 g_J^2 J(J+1) \beta}{3}
            \end{equation}
            と表され、これは\textbf{キュリーの法則}と呼ばれる。
            磁化率は温度に反比例し、高温の領域で磁化率が小さくなることが分かり、これは直感に従う結果である。

        \section{反磁性}
            閉殻構造を持つような原子は、常磁性ではなく\textbf{反磁性}と呼ばれる性質を持つ。
            これは原子全体で見た時に磁気モーメントがゼロになるためで、それにより\textbf{ラーモアの反磁性}と呼ばれる現象が起こる。
            そもそも反磁性は\textbf{レンツの法則}という、ある変化に対してそれを打ち消す向きに磁化する性質から生まれる。
            つまり、磁化率が負の値を示すということである。
            磁化率の詳しい計算方法については省略するが、磁気モーメントの変化$\Delta \mu $を使用して、
            \begin{equation}
                \chi = \frac{M}{H} = \frac{\mu_0 N Z \Delta \mu}{B}
            \end{equation}
            と表され、これを\textbf{ランジュバン反磁性}と呼ぶこともある。


        \section{自由電子の磁性}
            自由電子は、\textbf{パウリ常磁性}と呼ばれるような常磁性の性質と、\textbf{ランダウ反磁性}と呼ばれる反磁性の性質を併せ持つ。


    \chapter{半導体}
        \section{半導体}
            \textbf{半導体}とは、電気伝導率が導体と絶縁体の中間のような値を持つようなもののことである。

        \section{真性半導体}
            

        \section{不純物半導体}
        \section{有効質量理論}
        \section{p-n接合}


    \chapter{超伝導}
        10章
    \chapter{プラズモンとか}

\end{document}
