\documentclass[a4paper]{jsreport}
\usepackage[utf8]{inputenc}
\usepackage{amsmath}
\usepackage{physics}

\title{固体物理学 - Solid State Physics}

\begin{document}

    \maketitle

    \tableofcontents

    \chapter{オーバービュー}
    \chapter{結晶構造}
        \section{格子ベクトル}
            ブラべ格子、格子ベクトルの表式\par
            結晶は膨大な数の原子が並んで形成されている。
            その原子は規則正しく並んでいる(とする)わけであるが、どの原子から見ても景色が同じような点の集まりを\textbf{ブラべ格子}という。
            表現が分かりづらいので少し噛み砕くと、どのような周期性を持って原子が並んでいるのかを判別するためにピックアップした原子群といったようなイメージである。
            基本的な固体物理学のアプローチとして、周期構造の中の一つ(\textbf{単位格子})で解析をし、それを周期的境界条件で実際のスケールに拡張する。
            そのプロセスの中で、ブラべ格子を考えることは非常に意義のあることである。\par
            その周期的境界条件をどのように表現するのか。
            ここで、\textbf{格子ベクトル}を\textbf{基本並進ベクトル}と呼ばれる$\boldsymbol{a_i}$を使用して、
            \begin{equation} \label{eq:21a}
                \boldsymbol{T} = n_1 \boldsymbol{a_1} + n_2 \boldsymbol{a_2} + n_3 \boldsymbol{a_3}
            \end{equation}
            と定義し、格子点の位置$\boldsymbol{r}$、$\boldsymbol{r'}$を用いて、
            \begin{equation}
                \boldsymbol{r'} = \boldsymbol{r} + \boldsymbol{T}
            \end{equation}
            となることとする。
            この格子ベクトルを使えば、周期性を表現することができる。

        \section{結晶面}
            結晶(格子)の面を表現するには\textbf{ミラー指数}と呼ばれる量を使用し、$(hkl)$というように表記する。
            ここでは、基本並進ベクトル$\boldsymbol{a_i}$を使って、その軸上での点$0$から$1$の範囲で指定し、その逆数をそれぞれ取ったものがそれらの点を繋いでできる面がミラー指数で表される結晶面となる。
            文章だけで説明することは難しいので、ここではこれぐらいの記述に留めておく。
            ちなみにミラー指数は、逆格子空間での記述に対応させており、表記として$(hkl) = (m_1m_2m_3)$と、逆格子ベクトルの係数を順に並べるように定義されている。

        \section{逆格子ベクトル}
            実空間での解析よりも、\textbf{逆格子空間}(波数空間)と呼ばれる、実空間と逆の次元を持つような空間での解析の方が見通しが良い。
            理由は様々であるが、運動量や分散との対応が分かりやすくなるなどがある。\par
            実空間での周期性を損なうことがないように、逆格子空間を定義していきたい。
            任意の物理量$f(\boldsymbol{r})$について、実格子の周期性から
            \begin{equation} \label{eq:23a}
                f(\boldsymbol{r} + \boldsymbol{T}) = f(\boldsymbol{r}) 
            \end{equation}
            が成り立つ。
            ここで周期性に相性が良いフーリエ級数展開を利用することにより、
            \begin{equation} \label{eq:23b}
                f(\boldsymbol{r}) = \sum_{m} A_m \exp(i\boldsymbol{G_m} \cdot \boldsymbol{r})
            \end{equation}
            と表現することができ、ここで現れた$\boldsymbol{G_m}$が逆格子ベクトルに当たるものである。
            平面波の表式と見比べると、波数$\boldsymbol{k}$に当たる部分が$\boldsymbol{G_m}$になっていることから、逆格子空間は波数空間と呼ばれることがあるのだろう。
            以上までの議論を踏まえて、\textbf{逆格子基本並進ベクトル}を
            \begin{equation}
                \boldsymbol{b_i} = 2 \pi \frac{\boldsymbol{a_j} \times \boldsymbol{a_k}}{\boldsymbol{a_i} \cdot \boldsymbol{a_j} \times \boldsymbol{a_k}}
            \end{equation}
            と定義し、これを用いて\textbf{逆格子ベクトル}を、
            \begin{equation}
                \boldsymbol{G_m} = m_1 \boldsymbol{b_1} + m_2 \boldsymbol{b_2} + m_3 \boldsymbol{b_3} 
            \end{equation}
            と表すことにする。
            これを実際に(\ref{eq:23b})式に代入してみると、(\ref{eq:23a})式を満たしていることが分かる。
            ここでは、
            \begin{equation} \label{eq:23f}
                \boldsymbol{a_i} \cdot \boldsymbol{b_j} = 2 \pi \delta_{i,j}
            \end{equation}
            という関係式が成り立っており、実空間の基本並進ベクトルと長さの情報に関するものだけが逆になったような、密接すぎる関係があることは見て取れる。
            実際、実格子空間と逆格子空間は、フーリエ変換と逆フーリエ変換により行き来することができる。
            フーリエ変換は、重ね合わさった波を波数ごとに分離する(波束 → 単一波)に対応し、逆フーリエ変換はその逆の操作に対応する。
            このことを考えると、(\ref{eq:23f})式で$2\pi$の因子が出てくることも直感的に理解することができる。

        \section{回折条件}
            位置$\boldsymbol{r}$にある体積素片での回折について考えたい。
            入射波と散乱波の波数をそれぞれ$\boldsymbol{k}$、$\boldsymbol{k'}$と表現することにすると、位相差による因子は、$\exp{i(\boldsymbol{k} - \boldsymbol{k'}) \cdot \boldsymbol{r}}$と表される。
            この位相差の因子を見て、強めあう条件は、波数差が周期性と一致するときであると言えるので、
            \begin{equation}
                \boldsymbol{k} - \boldsymbol{k'} = - \Delta k = \boldsymbol{G}
            \end{equation}
            となるときである。
            ちなみに、この$- \Delta k$は\textbf{散乱ベクトル}と呼ばれる量である。
            この散乱が\textbf{弾性散乱}である場合、式変形をしていくことで、
            \begin{equation} \label{eq:25b}
                2 \boldsymbol{k} \cdot \boldsymbol{G} = G^2
            \end{equation}
            という表式が現れる。
            ここまでの議論は、強め合う条件に関するもので、最後の表式は\textbf{ブラッグの条件}の別の表し方であると言える。

        \section{ブリルアンゾーン}
            実空間の単位格子に\textbf{ウィグナー・サイツ・セル}と呼ばれるものがある。
            これは各格子点との中間面に囲まれた領域のことで、格子内に原子をただ1つだけ含む最小の格子(単位格子)である。\par
            逆格子空間におけるウィグナー・サイツ・セルに当たるものが\textbf{ブリルアンゾーン}である。
            (\ref{eq:25b})式を利用して、両辺を$4$で割ることで、
            \begin{equation}
                \boldsymbol{k} \cdot \left(\frac{1}{2} \boldsymbol{G}\right) = \left(\frac{1}{2} G\right)^2
            \end{equation}
            という表式が成り立ち、これがブリルアンゾーンを表す。
            このブリルアンゾーンは、中間面によって囲まれる空間の取り方に任意性があり、体積が小さい順に\textbf{第一ブリルアンゾーン}、\textbf{第二ブリルアンゾーン}、、と呼ばれる。
            もちろんのこと、境界は(\ref{eq:25b})式を満たす、つまりブラッグの条件を満たすので回折現象により強め合う。

        \section{構造因子}
            散乱された波の振幅\textbf{(散乱振幅$F$)}は、その位置の電子密度$n(\boldsymbol{r})$に依存し、
            \begin{equation}
                F = N \int dV n(\boldsymbol{r}) \exp(-i\boldsymbol{G} \cdot \boldsymbol{r}) = NS_G
            \end{equation}
            と表される。
            ちなみに、振幅の2乗が強度に当たる。
            これより、電子間の相互作用などを踏まえて、導出過程は省略するが、
            \begin{equation}
                S_G = \sum_{j}f_j \exp(-i\boldsymbol{G} \cdot \boldsymbol{r_j})
            \end{equation}
            となり、この$S_G$を\textbf{構造因子}、$f_j$を\textbf{原子構造因子}と呼ぶ。
            $f_j$については、原子(の状態)ごとに決まった値を持つものである。
            この$S_G$は、結晶構造を如実に表すものであり、結晶にどこに光を当てるとどれほど散乱するのかなどをすることができ、例えば散乱強度が$0$になるようなパラメータでは弱め合っているということが分かる。

        
        

    \chapter{結合}
        \section{結合とは}
            原子はなぜ結合するのか。
            その理由はシンプルで、結合した方が安定するからである。
            それを定量的に見ていきたいのだが、原子が完全に自由な時と、結合して結晶になっている時のエネルギー差を\textbf{結合エネルギー}と呼ぶ。
            この結合エネルギーが大きいほど、結合しやすいと言える。\par
            2原子間の相互作用を表すポテンシャルエネルギー$V(R)$を使って、
            \begin{equation}
                F(R) = - \frac{dV(R)}{dR}
            \end{equation}
            は\textbf{原子間力}を与える。
            このポテンシャルエネルギー$V(R)$は、ある$R_0$で極小値を取り、その$R_0$が安定の位置となり、そのときの原子間力は$0$になる。\par

        \section{分子結合}
            2つの水素原子の結合について考える。
            それぞれの原子の波動関数を$\varphi_1$, $\varphi_2$と表現することにすると、分子の波動関数は、
            \begin{equation}
                    \phi = C_1 \varphi_1 + C_2 \varphi_2
            \end{equation}
            と一次結合で表される。
            この波動関数のエネルギー準位を考えると、元のエネルギー準位よりも低い準位(安定的で結合に寄与する価電子帯)と高い準位(非安定的で電気伝導に寄与する伝導帯)の2つに分裂する。
            前者を\textbf{結合性軌道}、後者を\textbf{反結合性軌道}という。

        \section{イオン結合}
        \section{共有結合}
        \section{金属結合}
        \section{ファン・デル・ワールス結合}
        \section{水素結合}
        
    \chapter{フォノン}
        p.79~
        \section{}
        \section{}
        \section{}
        \section{}
        \section{}
        \section{}
        \section{}
    \section{}
        
    \chapter{自由電子気体}
        6章
    \chapter{エネルギーバンド}
        7章
    \chapter{フェルミエネルギー}
        9章
    \chapter{半導体}
        8章
    \chapter{超伝導}
        10章
    \chapter{磁性}
        11-13章
    \chapter{名前決めてない}
        14章
    \chapter{光学的性質}
        15章
    \chapter{電気的性質}
        16章
    
    % ここから先は検討中(一旦ここまでで良いのでは??)
            
\end{document}